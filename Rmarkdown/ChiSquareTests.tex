% Options for packages loaded elsewhere
\PassOptionsToPackage{unicode}{hyperref}
\PassOptionsToPackage{hyphens}{url}
%
\documentclass[
  ignorenonframetext,
]{beamer}
\usepackage{pgfpages}
\setbeamertemplate{caption}[numbered]
\setbeamertemplate{caption label separator}{: }
\setbeamercolor{caption name}{fg=normal text.fg}
\beamertemplatenavigationsymbolsempty
% Prevent slide breaks in the middle of a paragraph
\widowpenalties 1 10000
\raggedbottom
\setbeamertemplate{part page}{
  \centering
  \begin{beamercolorbox}[sep=16pt,center]{part title}
    \usebeamerfont{part title}\insertpart\par
  \end{beamercolorbox}
}
\setbeamertemplate{section page}{
  \centering
  \begin{beamercolorbox}[sep=12pt,center]{part title}
    \usebeamerfont{section title}\insertsection\par
  \end{beamercolorbox}
}
\setbeamertemplate{subsection page}{
  \centering
  \begin{beamercolorbox}[sep=8pt,center]{part title}
    \usebeamerfont{subsection title}\insertsubsection\par
  \end{beamercolorbox}
}
\AtBeginPart{
  \frame{\partpage}
}
\AtBeginSection{
  \ifbibliography
  \else
    \frame{\sectionpage}
  \fi
}
\AtBeginSubsection{
  \frame{\subsectionpage}
}
\usepackage{amsmath,amssymb}
\usepackage{iftex}
\ifPDFTeX
  \usepackage[T1]{fontenc}
  \usepackage[utf8]{inputenc}
  \usepackage{textcomp} % provide euro and other symbols
\else % if luatex or xetex
  \usepackage{unicode-math} % this also loads fontspec
  \defaultfontfeatures{Scale=MatchLowercase}
  \defaultfontfeatures[\rmfamily]{Ligatures=TeX,Scale=1}
\fi
\usepackage{lmodern}
\usetheme[]{CambridgeUS}
\usecolortheme{seagull}
\usefonttheme{professionalfonts}
\ifPDFTeX\else
  % xetex/luatex font selection
\fi
% Use upquote if available, for straight quotes in verbatim environments
\IfFileExists{upquote.sty}{\usepackage{upquote}}{}
\IfFileExists{microtype.sty}{% use microtype if available
  \usepackage[]{microtype}
  \UseMicrotypeSet[protrusion]{basicmath} % disable protrusion for tt fonts
}{}
\makeatletter
\@ifundefined{KOMAClassName}{% if non-KOMA class
  \IfFileExists{parskip.sty}{%
    \usepackage{parskip}
  }{% else
    \setlength{\parindent}{0pt}
    \setlength{\parskip}{6pt plus 2pt minus 1pt}}
}{% if KOMA class
  \KOMAoptions{parskip=half}}
\makeatother
\usepackage{xcolor}
\newif\ifbibliography
\usepackage{color}
\usepackage{fancyvrb}
\newcommand{\VerbBar}{|}
\newcommand{\VERB}{\Verb[commandchars=\\\{\}]}
\DefineVerbatimEnvironment{Highlighting}{Verbatim}{commandchars=\\\{\}}
% Add ',fontsize=\small' for more characters per line
\usepackage{framed}
\definecolor{shadecolor}{RGB}{248,248,248}
\newenvironment{Shaded}{\begin{snugshade}}{\end{snugshade}}
\newcommand{\AlertTok}[1]{\textcolor[rgb]{0.94,0.16,0.16}{#1}}
\newcommand{\AnnotationTok}[1]{\textcolor[rgb]{0.56,0.35,0.01}{\textbf{\textit{#1}}}}
\newcommand{\AttributeTok}[1]{\textcolor[rgb]{0.13,0.29,0.53}{#1}}
\newcommand{\BaseNTok}[1]{\textcolor[rgb]{0.00,0.00,0.81}{#1}}
\newcommand{\BuiltInTok}[1]{#1}
\newcommand{\CharTok}[1]{\textcolor[rgb]{0.31,0.60,0.02}{#1}}
\newcommand{\CommentTok}[1]{\textcolor[rgb]{0.56,0.35,0.01}{\textit{#1}}}
\newcommand{\CommentVarTok}[1]{\textcolor[rgb]{0.56,0.35,0.01}{\textbf{\textit{#1}}}}
\newcommand{\ConstantTok}[1]{\textcolor[rgb]{0.56,0.35,0.01}{#1}}
\newcommand{\ControlFlowTok}[1]{\textcolor[rgb]{0.13,0.29,0.53}{\textbf{#1}}}
\newcommand{\DataTypeTok}[1]{\textcolor[rgb]{0.13,0.29,0.53}{#1}}
\newcommand{\DecValTok}[1]{\textcolor[rgb]{0.00,0.00,0.81}{#1}}
\newcommand{\DocumentationTok}[1]{\textcolor[rgb]{0.56,0.35,0.01}{\textbf{\textit{#1}}}}
\newcommand{\ErrorTok}[1]{\textcolor[rgb]{0.64,0.00,0.00}{\textbf{#1}}}
\newcommand{\ExtensionTok}[1]{#1}
\newcommand{\FloatTok}[1]{\textcolor[rgb]{0.00,0.00,0.81}{#1}}
\newcommand{\FunctionTok}[1]{\textcolor[rgb]{0.13,0.29,0.53}{\textbf{#1}}}
\newcommand{\ImportTok}[1]{#1}
\newcommand{\InformationTok}[1]{\textcolor[rgb]{0.56,0.35,0.01}{\textbf{\textit{#1}}}}
\newcommand{\KeywordTok}[1]{\textcolor[rgb]{0.13,0.29,0.53}{\textbf{#1}}}
\newcommand{\NormalTok}[1]{#1}
\newcommand{\OperatorTok}[1]{\textcolor[rgb]{0.81,0.36,0.00}{\textbf{#1}}}
\newcommand{\OtherTok}[1]{\textcolor[rgb]{0.56,0.35,0.01}{#1}}
\newcommand{\PreprocessorTok}[1]{\textcolor[rgb]{0.56,0.35,0.01}{\textit{#1}}}
\newcommand{\RegionMarkerTok}[1]{#1}
\newcommand{\SpecialCharTok}[1]{\textcolor[rgb]{0.81,0.36,0.00}{\textbf{#1}}}
\newcommand{\SpecialStringTok}[1]{\textcolor[rgb]{0.31,0.60,0.02}{#1}}
\newcommand{\StringTok}[1]{\textcolor[rgb]{0.31,0.60,0.02}{#1}}
\newcommand{\VariableTok}[1]{\textcolor[rgb]{0.00,0.00,0.00}{#1}}
\newcommand{\VerbatimStringTok}[1]{\textcolor[rgb]{0.31,0.60,0.02}{#1}}
\newcommand{\WarningTok}[1]{\textcolor[rgb]{0.56,0.35,0.01}{\textbf{\textit{#1}}}}
\setlength{\emergencystretch}{3em} % prevent overfull lines
\providecommand{\tightlist}{%
  \setlength{\itemsep}{0pt}\setlength{\parskip}{0pt}}
\setcounter{secnumdepth}{-\maxdimen} % remove section numbering
\usepackage{placeins}
\usepackage{color}
\usepackage{bm}
\usepackage{amsmath}
\usepackage{algorithm}
\usepackage[]{algpseudocode}
\usepackage{tabularx}
\usepackage{multirow}
\usepackage[most]{tcolorbox}
\usepackage{tikz}
\usepackage{lipsum}
\usepackage{mathtools}
\usepackage{actuarialangle}
\ifLuaTeX
  \usepackage{selnolig}  % disable illegal ligatures
\fi
\IfFileExists{bookmark.sty}{\usepackage{bookmark}}{\usepackage{hyperref}}
\IfFileExists{xurl.sty}{\usepackage{xurl}}{} % add URL line breaks if available
\urlstyle{same}
\hypersetup{
  pdftitle={STT 3850 : Chi-Square Tests},
  pdfauthor={Fall 2023},
  hidelinks,
  pdfcreator={LaTeX via pandoc}}

\title{STT 3850 : Chi-Square Tests}
\author{Fall 2023}
\date{}
\institute{Appalachian State University}

\begin{document}
\frame{\titlepage}

\hypertarget{chi-square-goodness-of-fit-tests}{%
\section{Chi-Square Goodness-of-Fit
Tests}\label{chi-square-goodness-of-fit-tests}}

\begin{frame}{Background}
\protect\hypertarget{background}{}
\begin{itemize}
\item
  Many statistical procedures require knowledge of the population from
  which the sample is taken. For example, using Student's
  \(t\)-distribution for testing a hypothesis or constructing a
  confidence interval for \(\mu\) assumes that the parent population is
  normal.
\item
  \textbf{Goodness-of-fit} (GOF) procedures are presented that will help
  to identify the distribution of the population from which the sample
  is drawn.
\item
  The null hypothesis in a goodness-of-fit test is a statement about the
  form of the cumulative distribution. When all the parameters in the
  null hypothesis are specified, the hypothesis is called
  \textbf{simple}.
\item
  Recall that in the event the null hypothesis does not completely
  specify all of the parameters of the distribution, the hypothesis is
  said to be \textbf{composite}.
\end{itemize}
\end{frame}

\begin{frame}{Background}
\protect\hypertarget{background-1}{}
\begin{itemize}
\item
  Goodness-of-fit tests are typically used when the form of the
  population is in question. In contrast to most of the statistical
  procedures discussed so far, where the goal has been to
  \textbf{reject} the null hypothesis, in a GOF test one hopes to
  \textbf{retain} the null hypothesis.
\item
  Given a single random sample of size \(n\) from an unknown population
  \(F_X\), one may wish to test the hypothesis that \(F_X\) has some
  known distribution \(F_0(x)\) for all \(x\).
\end{itemize}
\end{frame}

\begin{frame}[fragile]{Background}
\protect\hypertarget{background-2}{}
\begin{itemize}
\item
  For example, using the data frame \texttt{SOCCER} from the
  \texttt{PASWR2} package, is it reasonable to assume the number of
  goals scored during regulation time for the 232 soccer matches has a
  Poisson distribution with \(\lambda=2.5\)?
\item
  Before applying the chi-square goodness-of-fit test, the data must be
  grouped according to some scheme to form \(k\) mutually exclusive
  categories. When the null hypothesis completely specifies the
  population, the probability that a random observation will fall into
  each of the chosen or fixed categories can be computed.
\end{itemize}
\end{frame}

\begin{frame}{Background}
\protect\hypertarget{background-3}{}
\begin{itemize}
\item
  Once the probabilities for a data point to fall into each of the
  chosen or fixed categories is computed, multiplying the probabilities
  by \(n\) produces the expected counts for each category under the null
  distribution.
\item
  If the null hypothesis is true, the differences between the counts
  observed in the \(k\) categories and the counts expected in the \(k\)
  categories should be small.
\end{itemize}
\end{frame}

\begin{frame}{Background}
\protect\hypertarget{background-4}{}
\begin{itemize}
\tightlist
\item
  The test criterion for testing
  \(H_0: F_X(x) = F_0(x) \text{ for all } x\) against the alternative
  \(H_1: F_X(x) \ne F_0(x) \text{ for some } x\) when the null
  hypothesis is completely specified is
\end{itemize}

\begin{equation}
\chi_{\text{obs}}^2=\sum_{i=1}^{k} \frac{(O_k - E_k)^2}{E_k},
\end{equation}

where \(\chi_\text{obs}^2\) is the sum of the squared deviations between
what is observed \((O_k)\) and what is expected \((E_k)\) in each of the
\(k\) categories divided by what is expected in each of the \(k\)
categories. Large values of \(\chi_\text{obs}^2\) occur when the
observed data are inconsistent with the null hypothesis and thus lead to
rejection of the null hypothesis. The exact distribution of
\(\chi_\text{obs}^2\) is very complicated; however, for large \(n\),
provided all expected categories are at least 5, \(\chi_\text{obs}^2\)
is distributed approximately \(\chi^2\) with \(k-1\) degrees of freedom.
\end{frame}

\begin{frame}{Background}
\protect\hypertarget{background-5}{}
\begin{itemize}
\tightlist
\item
  NOTE: When the null hypothesis is composite, that is, not all of the
  parameters are specified, the degrees of freedom for the random
  variable \(\chi_\text{obs}^2\) are reduced by one for each parameter
  that must be estimated.
\end{itemize}
\end{frame}

\begin{frame}[fragile]{Soccer Example}
\protect\hypertarget{soccer-example}{}
Test the hypothesis that the number of goals scored during regulation
time for the 232 soccer matches stored in the data frame \texttt{SOCCER}
has a Poisson \texttt{cdf} with \(\lambda=2.5\) with the chi-square
goodness-of-fit test and an \(\alpha\) level of 0.05. Produce a
histogram showing the number of observed goals scored during regulation
time and superimpose on the histogram the number of goals that are
expected to be made when the distribution of goals follows a Poisson
distribution with \(\lambda=2.5\).
\end{frame}

\begin{frame}[fragile]{Soccer Solution}
\protect\hypertarget{soccer-solution}{}
\begin{itemize}
\tightlist
\item
  Since the number of categories for a Poisson distribution is
  theoretically infinite, a table is first constructed of the observed
  number of goals to get an idea of reasonable categories.
\end{itemize}

\begin{Shaded}
\begin{Highlighting}[]
\FunctionTok{library}\NormalTok{(PASWR2)}
\FunctionTok{xtabs}\NormalTok{(}\SpecialCharTok{\textasciitilde{}}\NormalTok{goals, }\AttributeTok{data =}\NormalTok{ SOCCER)}
\end{Highlighting}
\end{Shaded}

\begin{verbatim}
goals
 0  1  2  3  4  5  6  7  8 
19 49 60 47 32 18  3  3  1 
\end{verbatim}
\end{frame}

\begin{frame}[fragile]{Soccer Solution}
\protect\hypertarget{soccer-solution-1}{}
Based on the table, a decision is made to create categories for 0, 1, 2,
3, 4, 5, and 6 or more goals. Under the null hypothesis that \(F_0(x)\)
is a Poisson distribution with \(\lambda=2.5\), the probabilities of
scoring 0, 1, 2, 3, 4, 5, and 6 or more goals are computed with
\texttt{R} as follows:

\begin{Shaded}
\begin{Highlighting}[]
\NormalTok{PX }\OtherTok{\textless{}{-}} \FunctionTok{c}\NormalTok{(}\FunctionTok{dpois}\NormalTok{(}\DecValTok{0}\SpecialCharTok{:}\DecValTok{5}\NormalTok{, }\FloatTok{2.5}\NormalTok{), }\FunctionTok{ppois}\NormalTok{(}\DecValTok{5}\NormalTok{, }\FloatTok{2.5}\NormalTok{, }\AttributeTok{lower =} \ConstantTok{FALSE}\NormalTok{))}
\NormalTok{PX[}\DecValTok{1}\SpecialCharTok{:}\DecValTok{4}\NormalTok{] }\CommentTok{\# Probabilities for categories 0, 1, 2, 3}
\end{Highlighting}
\end{Shaded}

\begin{verbatim}
[1] 0.0820850 0.2052125 0.2565156 0.2137630
\end{verbatim}

\begin{Shaded}
\begin{Highlighting}[]
\NormalTok{PX[}\DecValTok{4}\SpecialCharTok{:}\DecValTok{6}\NormalTok{] }\CommentTok{\# Probabilities for categories 4, 5, and 6 or more}
\end{Highlighting}
\end{Shaded}

\begin{verbatim}
[1] 0.21376302 0.13360189 0.06680094
\end{verbatim}
\end{frame}

\begin{frame}[fragile]{Soccer Solution}
\protect\hypertarget{soccer-solution-2}{}
\begin{tcolorbox}
Since there were a total of $n=232$ soccer games, the expected
number of goals for the six categories is simply $232 \times
\tt{PX}$.
\end{tcolorbox}

\begin{Shaded}
\begin{Highlighting}[]
\NormalTok{EX }\OtherTok{\textless{}{-}} \DecValTok{232}\SpecialCharTok{*}\NormalTok{PX}
\NormalTok{OB }\OtherTok{\textless{}{-}} \FunctionTok{c}\NormalTok{(}\FunctionTok{as.vector}\NormalTok{(}\FunctionTok{xtabs}\NormalTok{(}\SpecialCharTok{\textasciitilde{}}\NormalTok{goals, }\AttributeTok{data =}\NormalTok{ SOCCER)[}\DecValTok{1}\SpecialCharTok{:}\DecValTok{6}\NormalTok{]), }
        \FunctionTok{sum}\NormalTok{(}\FunctionTok{xtabs}\NormalTok{(}\SpecialCharTok{\textasciitilde{}}\NormalTok{goals, }\AttributeTok{data =}\NormalTok{ SOCCER)[}\DecValTok{7}\SpecialCharTok{:}\DecValTok{9}\NormalTok{]))}
\NormalTok{OB}
\end{Highlighting}
\end{Shaded}

\begin{verbatim}
[1] 19 49 60 47 32 18  7
\end{verbatim}

\begin{Shaded}
\begin{Highlighting}[]
\NormalTok{ans }\OtherTok{\textless{}{-}} \FunctionTok{cbind}\NormalTok{(PX, EX, OB)}
\FunctionTok{row.names}\NormalTok{(ans) }\OtherTok{\textless{}{-}} \FunctionTok{c}\NormalTok{(}\StringTok{" X=0"}\NormalTok{,}\StringTok{" X=1"}\NormalTok{,}\StringTok{" X=2"}\NormalTok{, }
                    \StringTok{" X=3"}\NormalTok{,}\StringTok{" X=4"}\NormalTok{,}\StringTok{" X=5"}\NormalTok{,}\StringTok{"X\textgreater{}=6"}\NormalTok{)}
\end{Highlighting}
\end{Shaded}
\end{frame}

\begin{frame}[fragile]{Soccer Solution}
\protect\hypertarget{soccer-solution-3}{}
\begin{Shaded}
\begin{Highlighting}[]
\NormalTok{ans}
\end{Highlighting}
\end{Shaded}

\begin{verbatim}
             PX        EX OB
 X=0 0.08208500 19.043720 19
 X=1 0.20521250 47.609299 49
 X=2 0.25651562 59.511624 60
 X=3 0.21376302 49.593020 47
 X=4 0.13360189 30.995638 32
 X=5 0.06680094 15.497819 18
X>=6 0.04202104  9.748881  7
\end{verbatim}
\end{frame}

\begin{frame}[fragile]{Soccer Solution}
\protect\hypertarget{soccer-solution-4}{}
The null and alternative hypotheses for using the chi-square
goodness-of-fit test to test the hypothesis that the number of goals
scored during regulation time for the 232 soccer matches stored in the
data frame \texttt{SOCCER} has a Poisson \texttt{cdf} with
\(\lambda=2.5\) are

\begin{align*}
        H_0&: F_X(x) = F_0(x) \sim Pois(\lambda=2.5)\text{ for all } x \text{ versus
        }\\
        H_1&:  F_X(x) \ne F_0(x) \text{ for some } x.
\end{align*}
\end{frame}

\begin{frame}[fragile]{Soccer Solution}
\protect\hypertarget{soccer-solution-5}{}
\begin{itemize}
\item
  The test statistic chosen is \(\chi_{\text{obs}}^2.\)
\item
  Reject if \(\chi^2_{\text{obs}}>\chi^2_{1-\alpha;k-1}\).
\end{itemize}

\begin{Shaded}
\begin{Highlighting}[]
\NormalTok{chi.obs }\OtherTok{\textless{}{-}} \FunctionTok{sum}\NormalTok{((OB}\SpecialCharTok{{-}}\NormalTok{EX)}\SpecialCharTok{\^{}}\DecValTok{2}\SpecialCharTok{/}\NormalTok{EX)}
\NormalTok{chi.obs}
\end{Highlighting}
\end{Shaded}

\begin{verbatim}
[1] 1.39194
\end{verbatim}
\end{frame}

\begin{frame}[fragile]{Soccer Solution}
\protect\hypertarget{soccer-solution-6}{}
\begin{Shaded}
\begin{Highlighting}[]
\FunctionTok{chisq.test}\NormalTok{(}\AttributeTok{x =}\NormalTok{ OB, }\AttributeTok{p =}\NormalTok{ PX)}
\end{Highlighting}
\end{Shaded}

\begin{verbatim}

    Chi-squared test for given probabilities

data:  OB
X-squared = 1.3919, df = 6, p-value = 0.9663
\end{verbatim}
\end{frame}

\begin{frame}[fragile]{Soccer Solution}
\protect\hypertarget{soccer-solution-7}{}
\(1.3919402=\chi^2_{\text{obs}}\overset{?}{>}\chi^2_{0.95;6}=12.5915872\).

The \(p\)-value is 0.9663469.

\begin{Shaded}
\begin{Highlighting}[]
\NormalTok{p.val }\OtherTok{\textless{}{-}} \FunctionTok{pchisq}\NormalTok{(chi.obs, }\DecValTok{7{-}1}\NormalTok{, }\AttributeTok{lower =} \ConstantTok{FALSE}\NormalTok{)}
\NormalTok{p.val}
\end{Highlighting}
\end{Shaded}

\begin{verbatim}
[1] 0.9663469
\end{verbatim}
\end{frame}

\begin{frame}{Soccer Solution}
\protect\hypertarget{soccer-solution-8}{}
\begin{itemize}
\item
  Since \(\chi^2_{\text{obs}}= 1.3919402\) is not greater than
  \(\chi^2_{0.95;6}=12.5915872\), fail to reject \(H_0\).
\item
  Since the \(p\)-value = 0.9663469 is greater than 0.05, fail to reject
  \(H_0\).
\end{itemize}
\end{frame}

\begin{frame}{Soccer Solution}
\protect\hypertarget{soccer-solution-9}{}
\begin{tcolorbox}
\textbf{English Conclusion:} There is no evidence to suggest that the true \textbf{cdf} does not equal the Poisson 
distribution with $\lambda=2.5$ for at least one $x$.
\end{tcolorbox}
\end{frame}

\begin{frame}[fragile]{Soccer Solution}
\protect\hypertarget{soccer-solution-10}{}
The following code can be used to create a histogram with superimposed
expected goals.

\begin{Shaded}
\begin{Highlighting}[]
\FunctionTok{hist}\NormalTok{(SOCCER}\SpecialCharTok{$}\NormalTok{goals, }\AttributeTok{breaks =} \FunctionTok{c}\NormalTok{((}\SpecialCharTok{{-}}\FloatTok{0.5} \SpecialCharTok{+} \DecValTok{0}\NormalTok{)}\SpecialCharTok{:}\NormalTok{(}\DecValTok{8} \SpecialCharTok{+} \FloatTok{0.5}\NormalTok{)), }
     \AttributeTok{col =} \StringTok{"lightblue"}\NormalTok{, }
     \AttributeTok{xlab =} \StringTok{"Goals scored"}\NormalTok{, }\AttributeTok{ylab =} \StringTok{""}\NormalTok{, }
     \AttributeTok{freq =} \ConstantTok{TRUE}\NormalTok{, }\AttributeTok{main =} \StringTok{""}\NormalTok{)}
\NormalTok{x }\OtherTok{\textless{}{-}} \DecValTok{0}\SpecialCharTok{:}\DecValTok{8}
\NormalTok{fx }\OtherTok{\textless{}{-}}\NormalTok{ (}\FunctionTok{dpois}\NormalTok{(}\DecValTok{0}\SpecialCharTok{:}\DecValTok{8}\NormalTok{, }\AttributeTok{lambda =} \FloatTok{2.5}\NormalTok{))}\SpecialCharTok{*}\DecValTok{232}
\FunctionTok{lines}\NormalTok{(x, fx, }\AttributeTok{type =} \StringTok{"h"}\NormalTok{)}
\FunctionTok{lines}\NormalTok{(x, fx, }\AttributeTok{type =} \StringTok{"p"}\NormalTok{, }\AttributeTok{pch =} \DecValTok{16}\NormalTok{)}
\end{Highlighting}
\end{Shaded}
\end{frame}

\begin{frame}{Soccer Solution}
\protect\hypertarget{soccer-solution-11}{}
\begin{center}\includegraphics[width=0.85\linewidth,height=0.65\textheight]{ChiSquareTests_files/figure-beamer/unnamed-chunk-9-1} \end{center}
\end{frame}

\begin{frame}[fragile]{All Parameters Known}
\protect\hypertarget{all-parameters-known}{}
\begin{itemize}
\tightlist
\item
  Bansley et al.~(1992) investigated the relationship between month of
  birth and achievement in sport. Birth dates were collected for players
  in teams competing in the 1990 World Cup soccer games.
\end{itemize}

\begin{Shaded}
\begin{Highlighting}[]
\NormalTok{Observed }\OtherTok{\textless{}{-}} \FunctionTok{c}\NormalTok{(}\DecValTok{150}\NormalTok{, }\DecValTok{138}\NormalTok{, }\DecValTok{140}\NormalTok{, }\DecValTok{100}\NormalTok{)}
\FunctionTok{names}\NormalTok{(Observed) }\OtherTok{\textless{}{-}} \FunctionTok{c}\NormalTok{(}\StringTok{"Aug{-}Oct"}\NormalTok{, }\StringTok{"Nov{-}Jan"}\NormalTok{, }
                     \StringTok{"Feb{-}April"}\NormalTok{, }\StringTok{"May{-}July"}\NormalTok{)}
\NormalTok{Observed}
\end{Highlighting}
\end{Shaded}

\begin{verbatim}
  Aug-Oct   Nov-Jan Feb-April  May-July 
      150       138       140       100 
\end{verbatim}
\end{frame}

\begin{frame}{All Parameters Known}
\protect\hypertarget{all-parameters-known-1}{}
We wish to test whether these data are consistent with the hypothesis
that birthdays of soccer players are uniformly distributed across the
four quarters of the year. Let \(P_i\) denote the probability of a birth
occurring in the \(i^{th}\) quarter; the hypotheses are as follows:

\(H_0: p_1=\frac{1}{4}, p_2=\frac{1}{4}, p_3=\frac{1}{4}, p_4=\frac{1}{4}\)
versus \(H_A: p_i \neq \frac{1}{4}\) for at least one \(i\).

There were a total of \(n = 528\) players considered for this study, so
the expected count for each quarter is \(528/4 = 132\).
\end{frame}

\begin{frame}[fragile]{All Parameters Known}
\protect\hypertarget{all-parameters-known-2}{}
\(\chi^2_{obs} = \sum_{i=1}^k\frac{(O_i - E_i)^2}{E_i} = \frac{(150 - 132)^2}{132} + \frac{(138 - 132)^2}{132} + \frac{(140 - 132)^2}{132} + \frac{(100 - 132)^2}{132} = 10.97\)

\begin{Shaded}
\begin{Highlighting}[]
\NormalTok{(chi\_obs }\OtherTok{\textless{}{-}} \FunctionTok{sum}\NormalTok{((Observed }\SpecialCharTok{{-}} \DecValTok{132}\NormalTok{)}\SpecialCharTok{\^{}}\DecValTok{2}\SpecialCharTok{/}\DecValTok{132}\NormalTok{))}
\end{Highlighting}
\end{Shaded}

\begin{verbatim}
[1] 10.9697
\end{verbatim}

\begin{Shaded}
\begin{Highlighting}[]
\CommentTok{\# Or}
\FunctionTok{chisq.test}\NormalTok{(Observed, }\AttributeTok{p =} \FunctionTok{c}\NormalTok{(}\DecValTok{1}\SpecialCharTok{/}\DecValTok{4}\NormalTok{, }\DecValTok{1}\SpecialCharTok{/}\DecValTok{4}\NormalTok{, }\DecValTok{1}\SpecialCharTok{/}\DecValTok{4}\NormalTok{, }\DecValTok{1}\SpecialCharTok{/}\DecValTok{4}\NormalTok{))}\SpecialCharTok{$}\NormalTok{stat}
\end{Highlighting}
\end{Shaded}

\begin{verbatim}
X-squared 
  10.9697 
\end{verbatim}
\end{frame}

\begin{frame}[fragile]{All Parameters Known}
\protect\hypertarget{all-parameters-known-3}{}
\begin{Shaded}
\begin{Highlighting}[]
\FunctionTok{chisq.test}\NormalTok{(Observed, }\AttributeTok{p =} \FunctionTok{c}\NormalTok{(}\DecValTok{1}\SpecialCharTok{/}\DecValTok{4}\NormalTok{, }\DecValTok{1}\SpecialCharTok{/}\DecValTok{4}\NormalTok{, }\DecValTok{1}\SpecialCharTok{/}\DecValTok{4}\NormalTok{, }\DecValTok{1}\SpecialCharTok{/}\DecValTok{4}\NormalTok{)) }\OtherTok{{-}\textgreater{}}\NormalTok{ CST}
\NormalTok{CST}
\end{Highlighting}
\end{Shaded}

\begin{verbatim}

    Chi-squared test for given probabilities

data:  Observed
X-squared = 10.97, df = 3, p-value = 0.01189
\end{verbatim}

\begin{Shaded}
\begin{Highlighting}[]
\NormalTok{CST}\SpecialCharTok{$}\NormalTok{observed}
\end{Highlighting}
\end{Shaded}

\begin{verbatim}
  Aug-Oct   Nov-Jan Feb-April  May-July 
      150       138       140       100 
\end{verbatim}

\begin{Shaded}
\begin{Highlighting}[]
\NormalTok{CST}\SpecialCharTok{$}\NormalTok{expected}
\end{Highlighting}
\end{Shaded}

\begin{verbatim}
  Aug-Oct   Nov-Jan Feb-April  May-July 
      132       132       132       132 
\end{verbatim}
\end{frame}

\begin{frame}[fragile]{All Parameters Known}
\protect\hypertarget{all-parameters-known-4}{}
\begin{Shaded}
\begin{Highlighting}[]
\NormalTok{(pvalue }\OtherTok{\textless{}{-}} \FunctionTok{pchisq}\NormalTok{(CST}\SpecialCharTok{$}\NormalTok{stat, }\DecValTok{3}\NormalTok{, }\AttributeTok{lower =} \ConstantTok{FALSE}\NormalTok{))}
\end{Highlighting}
\end{Shaded}

\begin{verbatim}
 X-squared 
0.01189087 
\end{verbatim}

\begin{Shaded}
\begin{Highlighting}[]
\CommentTok{\# Or}
\NormalTok{CST}\SpecialCharTok{$}\NormalTok{p.value}
\end{Highlighting}
\end{Shaded}

\begin{verbatim}
[1] 0.01189087
\end{verbatim}
\end{frame}

\begin{frame}{All Parameters Known - Conclusion}
\protect\hypertarget{all-parameters-known---conclusion}{}
\begin{tcolorbox}
Given the $p-value$ of $0.012$ evidence suggests birthdays for World Cup soccer players are not uniformly distributed.
\end{tcolorbox}
\end{frame}

\begin{frame}{All Parameters Known - Example 2}
\protect\hypertarget{all-parameters-known---example-2}{}
Suppose you draw 100 numbers at random from an unknown distribution.
Thirty values fall in the interval \((0, 0.25]\), 30 fall in
\((0.25, 0.75]\), 22 fall in \((0.75, 1.25]\), and the rest fall in
\((1.25, \infty]\). Your friend claims that the distribution is
exponential with parameter \(\lambda = 1\). Do you believe her?

\begin{itemize}
\tightlist
\item
  A random variable \(X\) has the exponential distribution with
  parameter \(\lambda > 0\) if its \textbf{pdf} is
\end{itemize}

\[f(x) = \lambda e^{-\lambda x},\quad x \geq 0.\]
\end{frame}

\begin{frame}{All Parameters Known - Example 2}
\protect\hypertarget{all-parameters-known---example-2-1}{}
We wish to test the following:

\begin{tcolorbox}
$H_0:$ The data are from an exponential distribution with $\lambda = 1$.

$H_A:$ The data are not from an exponential distribution with $\lambda = 1$.
\end{tcolorbox}
\end{frame}

\begin{frame}{All Parameters Known - Example 2}
\protect\hypertarget{all-parameters-known---example-2-2}{}
Given \(X \sim \text{Exp}(\lambda = 1)\). The probabilities for each
interval are as follows:

\(p_1 = P(0 \leq X \leq 0.25)=\int_0^{0.25}e^{-x}\,dx =0.2211992\)

\(p_2 = P(0.25 \leq X \leq 0.75)=\int_{0.25}^{0.75}e^{-x}\,dx =0.3064342\)

\(p_3 = P(0.75 \leq X \leq 1.25)=\int_{0.75}^{1.25}e^{-x}\,dx =0.1858618\)

\(p_4 = P(1.25 \leq X \leq \infty)=\int_{1.25}^{\infty}e^{-x}\,dx =0.2865048\)
\end{frame}

\begin{frame}[fragile]{All Parameters Known - Example 2}
\protect\hypertarget{all-parameters-known---example-2-3}{}
\begin{Shaded}
\begin{Highlighting}[]
\NormalTok{p1 }\OtherTok{\textless{}{-}} \FunctionTok{pexp}\NormalTok{(}\FloatTok{0.25}\NormalTok{, }\DecValTok{1}\NormalTok{)}
\NormalTok{p2 }\OtherTok{\textless{}{-}} \FunctionTok{pexp}\NormalTok{(}\FloatTok{0.75}\NormalTok{, }\DecValTok{1}\NormalTok{) }\SpecialCharTok{{-}} \FunctionTok{pexp}\NormalTok{(}\FloatTok{0.25}\NormalTok{, }\DecValTok{1}\NormalTok{)}
\NormalTok{p3 }\OtherTok{\textless{}{-}} \FunctionTok{pexp}\NormalTok{(}\FloatTok{1.25}\NormalTok{, }\DecValTok{1}\NormalTok{) }\SpecialCharTok{{-}} \FunctionTok{pexp}\NormalTok{(}\FloatTok{0.75}\NormalTok{, }\DecValTok{1}\NormalTok{)}
\NormalTok{p4 }\OtherTok{\textless{}{-}} \FunctionTok{pexp}\NormalTok{(}\FloatTok{1.25}\NormalTok{, }\DecValTok{1}\NormalTok{, }\AttributeTok{lower =} \ConstantTok{FALSE}\NormalTok{)}
\NormalTok{ps }\OtherTok{\textless{}{-}} \FunctionTok{c}\NormalTok{(p1, p2, p3, p4)}
\NormalTok{ps}
\end{Highlighting}
\end{Shaded}

\begin{verbatim}
[1] 0.2211992 0.3064342 0.1858618 0.2865048
\end{verbatim}
\end{frame}

\begin{frame}[fragile]{All Parameters Known - Example 2}
\protect\hypertarget{all-parameters-known---example-2-4}{}
\begin{Shaded}
\begin{Highlighting}[]
\NormalTok{EXP }\OtherTok{\textless{}{-}}\NormalTok{ ps}\SpecialCharTok{*}\DecValTok{100}
\NormalTok{EXP}
\end{Highlighting}
\end{Shaded}

\begin{verbatim}
[1] 22.11992 30.64342 18.58618 28.65048
\end{verbatim}

\begin{Shaded}
\begin{Highlighting}[]
\NormalTok{OBS }\OtherTok{\textless{}{-}} \FunctionTok{c}\NormalTok{(}\DecValTok{30}\NormalTok{, }\DecValTok{30}\NormalTok{, }\DecValTok{22}\NormalTok{, }\DecValTok{18}\NormalTok{)}
\NormalTok{test\_stat }\OtherTok{\textless{}{-}} \FunctionTok{sum}\NormalTok{((OBS }\SpecialCharTok{{-}}\NormalTok{ EXP)}\SpecialCharTok{\^{}}\DecValTok{2}\SpecialCharTok{/}\NormalTok{EXP)}
\NormalTok{test\_stat}
\end{Highlighting}
\end{Shaded}

\begin{verbatim}
[1] 7.406963
\end{verbatim}
\end{frame}

\begin{frame}[fragile]{All Parameters Known - Example 2}
\protect\hypertarget{all-parameters-known---example-2-5}{}
\begin{Shaded}
\begin{Highlighting}[]
\CommentTok{\# Another approach}
\FunctionTok{chisq.test}\NormalTok{(OBS, }\AttributeTok{p =}\NormalTok{ ps)}
\end{Highlighting}
\end{Shaded}

\begin{verbatim}

    Chi-squared test for given probabilities

data:  OBS
X-squared = 7.407, df = 3, p-value = 0.06
\end{verbatim}

\begin{Shaded}
\begin{Highlighting}[]
\NormalTok{pvalue }\OtherTok{\textless{}{-}} \FunctionTok{chisq.test}\NormalTok{(OBS, }\AttributeTok{p =}\NormalTok{ ps)}\SpecialCharTok{$}\NormalTok{p.value}
\NormalTok{pvalue}
\end{Highlighting}
\end{Shaded}

\begin{verbatim}
[1] 0.05999777
\end{verbatim}
\end{frame}

\begin{frame}{All Parameters Known - Example 2 - Conclusion}
\protect\hypertarget{all-parameters-known---example-2---conclusion}{}
\begin{tcolorbox}
If you test using $\alpha = 0.05$, you will fail to reject the null hypothesis since the $p-value$ $= 0.0599 > \alpha = 0.05$.  There is not convincing evidence that the data do not come from an Exp($\lambda = 1$).
\end{tcolorbox}
\end{frame}

\hypertarget{chi-square-tests-of-independence}{%
\section{Chi-Square Tests of
Independence}\label{chi-square-tests-of-independence}}

\begin{frame}[fragile]{Example}
\protect\hypertarget{example}{}
\begin{Shaded}
\begin{Highlighting}[]
\FunctionTok{library}\NormalTok{(PASWR2)}
\NormalTok{(}\FunctionTok{xtabs}\NormalTok{(}\SpecialCharTok{\textasciitilde{}}\NormalTok{sex }\SpecialCharTok{+}\NormalTok{ survived, }\AttributeTok{data =}\NormalTok{ TITANIC3) }\OtherTok{{-}\textgreater{}}\NormalTok{ T1)}
\end{Highlighting}
\end{Shaded}

\begin{verbatim}
        survived
sex        0   1
  female 127 339
  male   682 161
\end{verbatim}

\begin{Shaded}
\begin{Highlighting}[]
\FunctionTok{chisq.test}\NormalTok{(T1, }\AttributeTok{correct =} \ConstantTok{FALSE}\NormalTok{) }\OtherTok{{-}\textgreater{}}\NormalTok{ CST}
\NormalTok{CST}
\end{Highlighting}
\end{Shaded}

\begin{verbatim}

    Pearson's Chi-squared test

data:  T1
X-squared = 365.89, df = 1, p-value < 2.2e-16
\end{verbatim}
\end{frame}

\begin{frame}[fragile]{Example}
\protect\hypertarget{example-1}{}
\begin{Shaded}
\begin{Highlighting}[]
\NormalTok{(EXP }\OtherTok{\textless{}{-}}\NormalTok{ CST}\SpecialCharTok{$}\NormalTok{expected)}
\end{Highlighting}
\end{Shaded}

\begin{verbatim}
        survived
sex             0        1
  female 288.0015 177.9985
  male   520.9985 322.0015
\end{verbatim}

\begin{Shaded}
\begin{Highlighting}[]
\NormalTok{(OBS }\OtherTok{\textless{}{-}}\NormalTok{ CST}\SpecialCharTok{$}\NormalTok{observed)}
\end{Highlighting}
\end{Shaded}

\begin{verbatim}
        survived
sex        0   1
  female 127 339
  male   682 161
\end{verbatim}

\begin{Shaded}
\begin{Highlighting}[]
\NormalTok{(chi\_obs }\OtherTok{\textless{}{-}} \FunctionTok{sum}\NormalTok{((OBS }\SpecialCharTok{{-}}\NormalTok{ EXP)}\SpecialCharTok{\^{}}\DecValTok{2}\SpecialCharTok{/}\NormalTok{EXP))}
\end{Highlighting}
\end{Shaded}

\begin{verbatim}
[1] 365.8869
\end{verbatim}
\end{frame}

\hypertarget{chi-square-tests-of-homogeneity}{%
\section{Chi-Square Tests of
Homogeneity}\label{chi-square-tests-of-homogeneity}}

\begin{frame}[fragile]{Example}
\protect\hypertarget{example-2}{}
\begin{itemize}
\tightlist
\item
  Data will often come summarized in contingency tables.
\end{itemize}

\begin{Shaded}
\begin{Highlighting}[]
\NormalTok{DP }\OtherTok{\textless{}{-}} \FunctionTok{c}\NormalTok{(}\DecValTok{67}\NormalTok{, }\DecValTok{76}\NormalTok{, }\DecValTok{57}\NormalTok{, }\DecValTok{48}\NormalTok{, }\DecValTok{73}\NormalTok{, }\DecValTok{79}\NormalTok{)}
\NormalTok{MDP }\OtherTok{\textless{}{-}} \FunctionTok{matrix}\NormalTok{(}\AttributeTok{data =}\NormalTok{ DP, }\AttributeTok{nrow =} \DecValTok{2}\NormalTok{, }\AttributeTok{byrow =} \ConstantTok{TRUE}\NormalTok{)}
\FunctionTok{dimnames}\NormalTok{(MDP) }\OtherTok{\textless{}{-}} \FunctionTok{list}\NormalTok{(}\AttributeTok{Pop =} \FunctionTok{c}\NormalTok{(}\StringTok{"Drug"}\NormalTok{, }\StringTok{"Placebo"}\NormalTok{), }
    \AttributeTok{Status =} \FunctionTok{c}\NormalTok{(}\StringTok{"Improve"}\NormalTok{, }\StringTok{"No Change"}\NormalTok{, }\StringTok{"Worse"}\NormalTok{))}
\NormalTok{TDP }\OtherTok{\textless{}{-}} \FunctionTok{as.table}\NormalTok{(MDP)}
\NormalTok{TDP}
\end{Highlighting}
\end{Shaded}

\begin{verbatim}
         Status
Pop       Improve No Change Worse
  Drug         67        76    57
  Placebo      48        73    79
\end{verbatim}
\end{frame}

\begin{frame}[fragile]{Putting the data back in a tidy format}
\protect\hypertarget{putting-the-data-back-in-a-tidy-format}{}
\begin{Shaded}
\begin{Highlighting}[]
\FunctionTok{library}\NormalTok{(tidyverse)}
\NormalTok{NT }\OtherTok{\textless{}{-}}\NormalTok{ TDP }\SpecialCharTok{\%\textgreater{}\%} 
\NormalTok{  tibble}\SpecialCharTok{::}\FunctionTok{as\_tibble}\NormalTok{() }\SpecialCharTok{\%\textgreater{}\%} 
  \FunctionTok{uncount}\NormalTok{(n)}
\FunctionTok{head}\NormalTok{(NT, }\DecValTok{3}\NormalTok{)}
\end{Highlighting}
\end{Shaded}

\begin{verbatim}
# A tibble: 3 x 2
  Pop   Status 
  <chr> <chr>  
1 Drug  Improve
2 Drug  Improve
3 Drug  Improve
\end{verbatim}
\end{frame}

\end{document}
