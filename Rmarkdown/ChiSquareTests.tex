% Options for packages loaded elsewhere
\PassOptionsToPackage{unicode}{hyperref}
\PassOptionsToPackage{hyphens}{url}
%
\documentclass[
  ignorenonframetext,
]{beamer}
\usepackage{pgfpages}
\setbeamertemplate{caption}[numbered]
\setbeamertemplate{caption label separator}{: }
\setbeamercolor{caption name}{fg=normal text.fg}
\beamertemplatenavigationsymbolsempty
% Prevent slide breaks in the middle of a paragraph
\widowpenalties 1 10000
\raggedbottom
\setbeamertemplate{part page}{
  \centering
  \begin{beamercolorbox}[sep=16pt,center]{part title}
    \usebeamerfont{part title}\insertpart\par
  \end{beamercolorbox}
}
\setbeamertemplate{section page}{
  \centering
  \begin{beamercolorbox}[sep=12pt,center]{part title}
    \usebeamerfont{section title}\insertsection\par
  \end{beamercolorbox}
}
\setbeamertemplate{subsection page}{
  \centering
  \begin{beamercolorbox}[sep=8pt,center]{part title}
    \usebeamerfont{subsection title}\insertsubsection\par
  \end{beamercolorbox}
}
\AtBeginPart{
  \frame{\partpage}
}
\AtBeginSection{
  \ifbibliography
  \else
    \frame{\sectionpage}
  \fi
}
\AtBeginSubsection{
  \frame{\subsectionpage}
}
\usepackage{amsmath,amssymb}
\usepackage{iftex}
\ifPDFTeX
  \usepackage[T1]{fontenc}
  \usepackage[utf8]{inputenc}
  \usepackage{textcomp} % provide euro and other symbols
\else % if luatex or xetex
  \usepackage{unicode-math} % this also loads fontspec
  \defaultfontfeatures{Scale=MatchLowercase}
  \defaultfontfeatures[\rmfamily]{Ligatures=TeX,Scale=1}
\fi
\usepackage{lmodern}
\usetheme[]{CambridgeUS}
\usecolortheme{seagull}
\usefonttheme{professionalfonts}
\ifPDFTeX\else
  % xetex/luatex font selection
\fi
% Use upquote if available, for straight quotes in verbatim environments
\IfFileExists{upquote.sty}{\usepackage{upquote}}{}
\IfFileExists{microtype.sty}{% use microtype if available
  \usepackage[]{microtype}
  \UseMicrotypeSet[protrusion]{basicmath} % disable protrusion for tt fonts
}{}
\makeatletter
\@ifundefined{KOMAClassName}{% if non-KOMA class
  \IfFileExists{parskip.sty}{%
    \usepackage{parskip}
  }{% else
    \setlength{\parindent}{0pt}
    \setlength{\parskip}{6pt plus 2pt minus 1pt}}
}{% if KOMA class
  \KOMAoptions{parskip=half}}
\makeatother
\usepackage{xcolor}
\newif\ifbibliography
\usepackage{color}
\usepackage{fancyvrb}
\newcommand{\VerbBar}{|}
\newcommand{\VERB}{\Verb[commandchars=\\\{\}]}
\DefineVerbatimEnvironment{Highlighting}{Verbatim}{commandchars=\\\{\}}
% Add ',fontsize=\small' for more characters per line
\usepackage{framed}
\definecolor{shadecolor}{RGB}{248,248,248}
\newenvironment{Shaded}{\begin{snugshade}}{\end{snugshade}}
\newcommand{\AlertTok}[1]{\textcolor[rgb]{0.94,0.16,0.16}{#1}}
\newcommand{\AnnotationTok}[1]{\textcolor[rgb]{0.56,0.35,0.01}{\textbf{\textit{#1}}}}
\newcommand{\AttributeTok}[1]{\textcolor[rgb]{0.13,0.29,0.53}{#1}}
\newcommand{\BaseNTok}[1]{\textcolor[rgb]{0.00,0.00,0.81}{#1}}
\newcommand{\BuiltInTok}[1]{#1}
\newcommand{\CharTok}[1]{\textcolor[rgb]{0.31,0.60,0.02}{#1}}
\newcommand{\CommentTok}[1]{\textcolor[rgb]{0.56,0.35,0.01}{\textit{#1}}}
\newcommand{\CommentVarTok}[1]{\textcolor[rgb]{0.56,0.35,0.01}{\textbf{\textit{#1}}}}
\newcommand{\ConstantTok}[1]{\textcolor[rgb]{0.56,0.35,0.01}{#1}}
\newcommand{\ControlFlowTok}[1]{\textcolor[rgb]{0.13,0.29,0.53}{\textbf{#1}}}
\newcommand{\DataTypeTok}[1]{\textcolor[rgb]{0.13,0.29,0.53}{#1}}
\newcommand{\DecValTok}[1]{\textcolor[rgb]{0.00,0.00,0.81}{#1}}
\newcommand{\DocumentationTok}[1]{\textcolor[rgb]{0.56,0.35,0.01}{\textbf{\textit{#1}}}}
\newcommand{\ErrorTok}[1]{\textcolor[rgb]{0.64,0.00,0.00}{\textbf{#1}}}
\newcommand{\ExtensionTok}[1]{#1}
\newcommand{\FloatTok}[1]{\textcolor[rgb]{0.00,0.00,0.81}{#1}}
\newcommand{\FunctionTok}[1]{\textcolor[rgb]{0.13,0.29,0.53}{\textbf{#1}}}
\newcommand{\ImportTok}[1]{#1}
\newcommand{\InformationTok}[1]{\textcolor[rgb]{0.56,0.35,0.01}{\textbf{\textit{#1}}}}
\newcommand{\KeywordTok}[1]{\textcolor[rgb]{0.13,0.29,0.53}{\textbf{#1}}}
\newcommand{\NormalTok}[1]{#1}
\newcommand{\OperatorTok}[1]{\textcolor[rgb]{0.81,0.36,0.00}{\textbf{#1}}}
\newcommand{\OtherTok}[1]{\textcolor[rgb]{0.56,0.35,0.01}{#1}}
\newcommand{\PreprocessorTok}[1]{\textcolor[rgb]{0.56,0.35,0.01}{\textit{#1}}}
\newcommand{\RegionMarkerTok}[1]{#1}
\newcommand{\SpecialCharTok}[1]{\textcolor[rgb]{0.81,0.36,0.00}{\textbf{#1}}}
\newcommand{\SpecialStringTok}[1]{\textcolor[rgb]{0.31,0.60,0.02}{#1}}
\newcommand{\StringTok}[1]{\textcolor[rgb]{0.31,0.60,0.02}{#1}}
\newcommand{\VariableTok}[1]{\textcolor[rgb]{0.00,0.00,0.00}{#1}}
\newcommand{\VerbatimStringTok}[1]{\textcolor[rgb]{0.31,0.60,0.02}{#1}}
\newcommand{\WarningTok}[1]{\textcolor[rgb]{0.56,0.35,0.01}{\textbf{\textit{#1}}}}
\setlength{\emergencystretch}{3em} % prevent overfull lines
\providecommand{\tightlist}{%
  \setlength{\itemsep}{0pt}\setlength{\parskip}{0pt}}
\setcounter{secnumdepth}{-\maxdimen} % remove section numbering
\usepackage{placeins}
\usepackage{color}
\usepackage{bm}
\usepackage{amsmath}
\usepackage{algorithm}
\usepackage[]{algpseudocode}
\usepackage{tabularx}
\usepackage{multirow}
\usepackage[most]{tcolorbox}
\usepackage{tikz}
\usepackage{lipsum}
\usepackage{mathtools}
\usepackage{actuarialangle}
\ifLuaTeX
  \usepackage{selnolig}  % disable illegal ligatures
\fi
\IfFileExists{bookmark.sty}{\usepackage{bookmark}}{\usepackage{hyperref}}
\IfFileExists{xurl.sty}{\usepackage{xurl}}{} % add URL line breaks if available
\urlstyle{same}
\hypersetup{
  pdftitle={STT 3850 : Chi-Square Tests},
  pdfauthor={Fall 2023},
  hidelinks,
  pdfcreator={LaTeX via pandoc}}

\title{STT 3850 : Chi-Square Tests}
\author{Fall 2023}
\date{}
\institute{Appalachian State University}

\begin{document}
\frame{\titlepage}

\hypertarget{chi-square-goodness-of-fit-tests}{%
\section{Chi-Square Goodness-of-Fit
Tests}\label{chi-square-goodness-of-fit-tests}}

\begin{frame}[fragile]{All Parameters Known}
\protect\hypertarget{all-parameters-known}{}
\begin{itemize}
\tightlist
\item
  Bansley et al.~(1992) investigated the relationship between month of
  birth and achievement in sport. Birth dates were collected for players
  in teams competing in the 1990 World Cup soccer games.
\end{itemize}

\begin{Shaded}
\begin{Highlighting}[]
\NormalTok{Observed }\OtherTok{\textless{}{-}} \FunctionTok{c}\NormalTok{(}\DecValTok{150}\NormalTok{, }\DecValTok{138}\NormalTok{, }\DecValTok{140}\NormalTok{, }\DecValTok{100}\NormalTok{)}
\FunctionTok{names}\NormalTok{(Observed) }\OtherTok{\textless{}{-}} \FunctionTok{c}\NormalTok{(}\StringTok{"Aug{-}Oct"}\NormalTok{, }\StringTok{"Nov{-}Jan"}\NormalTok{, }
                     \StringTok{"Feb{-}April"}\NormalTok{, }\StringTok{"May{-}July"}\NormalTok{)}
\NormalTok{Observed}
\end{Highlighting}
\end{Shaded}

\begin{verbatim}
  Aug-Oct   Nov-Jan Feb-April  May-July 
      150       138       140       100 
\end{verbatim}
\end{frame}

\begin{frame}{All Parameters Known}
\protect\hypertarget{all-parameters-known-1}{}
We wish to test whether these data are consistent with the hypothesis
that birthdays of soccer players are uniformly distributed across the
four quarters of the year. Let \(P_i\) denote the probability of a birth
occurring in the \(i^{th}\) quarter; the hypotheses are as follows:

\(H_0: p_1=\frac{1}{4}, p_2=\frac{1}{4}, p_3=\frac{1}{4}, p_4=\frac{1}{4}\)
versus \(H_A: p_i \neq \frac{1}{4}\) for at least one \(i\).

There were a total of \(n = 528\) players considered for this study, so
the expected count for each quarter is \(528/4 = 132\).
\end{frame}

\begin{frame}[fragile]{All Parameters Known}
\protect\hypertarget{all-parameters-known-2}{}
\(\chi^2_{obs} = \sum_{i=1}^k\frac{(O_i - E_i)^2}{E_i} = \frac{(150 - 132)^2}{132} + \frac{(138 - 132)^2}{132} + \frac{(140 - 132)^2}{132} + \frac{(100 - 132)^2}{132} = 10.97\)

\begin{Shaded}
\begin{Highlighting}[]
\NormalTok{(chi\_obs }\OtherTok{\textless{}{-}} \FunctionTok{sum}\NormalTok{((Observed }\SpecialCharTok{{-}} \DecValTok{132}\NormalTok{)}\SpecialCharTok{\^{}}\DecValTok{2}\SpecialCharTok{/}\DecValTok{132}\NormalTok{))}
\end{Highlighting}
\end{Shaded}

\begin{verbatim}
[1] 10.9697
\end{verbatim}

\begin{Shaded}
\begin{Highlighting}[]
\CommentTok{\# Or}
\FunctionTok{chisq.test}\NormalTok{(Observed, }\AttributeTok{p =} \FunctionTok{c}\NormalTok{(}\DecValTok{1}\SpecialCharTok{/}\DecValTok{4}\NormalTok{, }\DecValTok{1}\SpecialCharTok{/}\DecValTok{4}\NormalTok{, }\DecValTok{1}\SpecialCharTok{/}\DecValTok{4}\NormalTok{, }\DecValTok{1}\SpecialCharTok{/}\DecValTok{4}\NormalTok{))}\SpecialCharTok{$}\NormalTok{stat}
\end{Highlighting}
\end{Shaded}

\begin{verbatim}
X-squared 
  10.9697 
\end{verbatim}
\end{frame}

\begin{frame}[fragile]{All Parameters Known}
\protect\hypertarget{all-parameters-known-3}{}
\begin{Shaded}
\begin{Highlighting}[]
\FunctionTok{chisq.test}\NormalTok{(Observed, }\AttributeTok{p =} \FunctionTok{c}\NormalTok{(}\DecValTok{1}\SpecialCharTok{/}\DecValTok{4}\NormalTok{, }\DecValTok{1}\SpecialCharTok{/}\DecValTok{4}\NormalTok{, }\DecValTok{1}\SpecialCharTok{/}\DecValTok{4}\NormalTok{, }\DecValTok{1}\SpecialCharTok{/}\DecValTok{4}\NormalTok{)) }\OtherTok{{-}\textgreater{}}\NormalTok{ CST}
\NormalTok{CST}
\end{Highlighting}
\end{Shaded}

\begin{verbatim}

    Chi-squared test for given probabilities

data:  Observed
X-squared = 10.97, df = 3, p-value = 0.01189
\end{verbatim}

\begin{Shaded}
\begin{Highlighting}[]
\NormalTok{CST}\SpecialCharTok{$}\NormalTok{observed}
\end{Highlighting}
\end{Shaded}

\begin{verbatim}
  Aug-Oct   Nov-Jan Feb-April  May-July 
      150       138       140       100 
\end{verbatim}

\begin{Shaded}
\begin{Highlighting}[]
\NormalTok{CST}\SpecialCharTok{$}\NormalTok{expected}
\end{Highlighting}
\end{Shaded}

\begin{verbatim}
  Aug-Oct   Nov-Jan Feb-April  May-July 
      132       132       132       132 
\end{verbatim}
\end{frame}

\begin{frame}[fragile]{All Parameters Known}
\protect\hypertarget{all-parameters-known-4}{}
\begin{Shaded}
\begin{Highlighting}[]
\NormalTok{(pvalue }\OtherTok{\textless{}{-}} \FunctionTok{pchisq}\NormalTok{(CST}\SpecialCharTok{$}\NormalTok{stat, }\DecValTok{3}\NormalTok{, }\AttributeTok{lower =} \ConstantTok{FALSE}\NormalTok{))}
\end{Highlighting}
\end{Shaded}

\begin{verbatim}
 X-squared 
0.01189087 
\end{verbatim}

\begin{Shaded}
\begin{Highlighting}[]
\CommentTok{\# Or}
\NormalTok{CST}\SpecialCharTok{$}\NormalTok{p.value}
\end{Highlighting}
\end{Shaded}

\begin{verbatim}
[1] 0.01189087
\end{verbatim}
\end{frame}

\begin{frame}{All Parameters Known - Conclusion}
\protect\hypertarget{all-parameters-known---conclusion}{}
\begin{tcolorbox}
Given the $p-value$ of $0.012$ evidence suggests birthdays for World Cup soccer players are not uniformly distributed.
\end{tcolorbox}
\end{frame}

\begin{frame}{All Parameters Known - Example 2}
\protect\hypertarget{all-parameters-known---example-2}{}
Suppose you draw 100 numbers at random from an unknown distribution.
Thirty values fall in the interval \((0, 0.25]\), 30 fall in
\((0.25, 0.75]\), 22 fall in \((0.75, 1.25]\), and the rest fall in
\((1.25, \infty]\). Your friend claims that the distribution is
exponential with parameter \(\lambda = 1\). Do you believe her?

\begin{itemize}
\tightlist
\item
  A random variable \(X\) has the exponential distribution with
  parameter \(\lambda > 0\) if its \textbf{pdf} is
\end{itemize}

\[f(x) = \lambda e^{-\lambda x},\quad x \geq 0.\]
\end{frame}

\begin{frame}{All Parameters Known - Example 2}
\protect\hypertarget{all-parameters-known---example-2-1}{}
We wish to test the following:

\begin{tcolorbox}
$H_0:$ The data are from an exponential distribution with $\lambda = 1$.

$H_A:$ The data are not from an exponential distribution with $\lambda = 1$.
\end{tcolorbox}
\end{frame}

\begin{frame}{All Parameters Known - Example 2}
\protect\hypertarget{all-parameters-known---example-2-2}{}
Given \(X \sim \text{Exp}(\lambda = 1)\). The probabilities for each
interval are as follows:

\(p_1 = P(0 \leq X \leq 0.25)=\int_0^{0.25}e^{-x}\,dx =0.2211992\)

\(p_2 = P(0.25 \leq X \leq 0.75)=\int_{0.25}^{0.75}e^{-x}\,dx =0.3064342\)

\(p_3 = P(0.75 \leq X \leq 1.25)=\int_{0.75}^{1.25}e^{-x}\,dx =0.1858618\)

\(p_4 = P(1.25 \leq X \leq \infty)=\int_{1.25}^{\infty}e^{-x}\,dx =0.2865048\)
\end{frame}

\begin{frame}[fragile]{All Parameters Known - Example 2}
\protect\hypertarget{all-parameters-known---example-2-3}{}
\begin{Shaded}
\begin{Highlighting}[]
\NormalTok{p1 }\OtherTok{\textless{}{-}} \FunctionTok{pexp}\NormalTok{(}\FloatTok{0.25}\NormalTok{, }\DecValTok{1}\NormalTok{)}
\NormalTok{p2 }\OtherTok{\textless{}{-}} \FunctionTok{pexp}\NormalTok{(}\FloatTok{0.75}\NormalTok{, }\DecValTok{1}\NormalTok{) }\SpecialCharTok{{-}} \FunctionTok{pexp}\NormalTok{(}\FloatTok{0.25}\NormalTok{, }\DecValTok{1}\NormalTok{)}
\NormalTok{p3 }\OtherTok{\textless{}{-}} \FunctionTok{pexp}\NormalTok{(}\FloatTok{1.25}\NormalTok{, }\DecValTok{1}\NormalTok{) }\SpecialCharTok{{-}} \FunctionTok{pexp}\NormalTok{(}\FloatTok{0.75}\NormalTok{, }\DecValTok{1}\NormalTok{)}
\NormalTok{p4 }\OtherTok{\textless{}{-}} \FunctionTok{pexp}\NormalTok{(}\FloatTok{1.25}\NormalTok{, }\DecValTok{1}\NormalTok{, }\AttributeTok{lower =} \ConstantTok{FALSE}\NormalTok{)}
\NormalTok{ps }\OtherTok{\textless{}{-}} \FunctionTok{c}\NormalTok{(p1, p2, p3, p4)}
\NormalTok{ps}
\end{Highlighting}
\end{Shaded}

\begin{verbatim}
[1] 0.2211992 0.3064342 0.1858618 0.2865048
\end{verbatim}
\end{frame}

\begin{frame}[fragile]{All Parameters Known - Example 2}
\protect\hypertarget{all-parameters-known---example-2-4}{}
\begin{Shaded}
\begin{Highlighting}[]
\NormalTok{EXP }\OtherTok{\textless{}{-}}\NormalTok{ ps}\SpecialCharTok{*}\DecValTok{100}
\NormalTok{EXP}
\end{Highlighting}
\end{Shaded}

\begin{verbatim}
[1] 22.11992 30.64342 18.58618 28.65048
\end{verbatim}

\begin{Shaded}
\begin{Highlighting}[]
\NormalTok{OBS }\OtherTok{\textless{}{-}} \FunctionTok{c}\NormalTok{(}\DecValTok{30}\NormalTok{, }\DecValTok{30}\NormalTok{, }\DecValTok{22}\NormalTok{, }\DecValTok{18}\NormalTok{)}
\NormalTok{test\_stat }\OtherTok{\textless{}{-}} \FunctionTok{sum}\NormalTok{((OBS }\SpecialCharTok{{-}}\NormalTok{ EXP)}\SpecialCharTok{\^{}}\DecValTok{2}\SpecialCharTok{/}\NormalTok{EXP)}
\NormalTok{test\_stat}
\end{Highlighting}
\end{Shaded}

\begin{verbatim}
[1] 7.406963
\end{verbatim}
\end{frame}

\begin{frame}[fragile]{All Parameters Known - Example 2}
\protect\hypertarget{all-parameters-known---example-2-5}{}
\begin{Shaded}
\begin{Highlighting}[]
\CommentTok{\# Another approach}
\FunctionTok{chisq.test}\NormalTok{(OBS, }\AttributeTok{p =}\NormalTok{ ps)}
\end{Highlighting}
\end{Shaded}

\begin{verbatim}

    Chi-squared test for given probabilities

data:  OBS
X-squared = 7.407, df = 3, p-value = 0.06
\end{verbatim}

\begin{Shaded}
\begin{Highlighting}[]
\NormalTok{pvalue }\OtherTok{\textless{}{-}} \FunctionTok{chisq.test}\NormalTok{(OBS, }\AttributeTok{p =}\NormalTok{ ps)}\SpecialCharTok{$}\NormalTok{p.value}
\NormalTok{pvalue}
\end{Highlighting}
\end{Shaded}

\begin{verbatim}
[1] 0.05999777
\end{verbatim}
\end{frame}

\begin{frame}{All Parameters Known - Example 2 - Conclusion}
\protect\hypertarget{all-parameters-known---example-2---conclusion}{}
\begin{tcolorbox}
If you test using $\alpha = 0.05$, you will fail to reject the null hypothesis since the $p-value$ $= 0.0599 > \alpha = 0.05$.  There is not convincing evidence that the data do not come from an Exp($\lambda = 1$).
\end{tcolorbox}
\end{frame}

\hypertarget{chi-square-tests-of-independence}{%
\section{Chi-Square Tests of
Independence}\label{chi-square-tests-of-independence}}

\begin{frame}[fragile]{Example}
\protect\hypertarget{example}{}
\begin{Shaded}
\begin{Highlighting}[]
\FunctionTok{library}\NormalTok{(PASWR2)}
\NormalTok{(}\FunctionTok{xtabs}\NormalTok{(}\SpecialCharTok{\textasciitilde{}}\NormalTok{sex }\SpecialCharTok{+}\NormalTok{ survived, }\AttributeTok{data =}\NormalTok{ TITANIC3) }\OtherTok{{-}\textgreater{}}\NormalTok{ T1)}
\end{Highlighting}
\end{Shaded}

\begin{verbatim}
        survived
sex        0   1
  female 127 339
  male   682 161
\end{verbatim}

\begin{Shaded}
\begin{Highlighting}[]
\FunctionTok{chisq.test}\NormalTok{(T1, }\AttributeTok{correct =} \ConstantTok{FALSE}\NormalTok{) }\OtherTok{{-}\textgreater{}}\NormalTok{ CST}
\NormalTok{CST}
\end{Highlighting}
\end{Shaded}

\begin{verbatim}

    Pearson's Chi-squared test

data:  T1
X-squared = 365.89, df = 1, p-value < 2.2e-16
\end{verbatim}
\end{frame}

\begin{frame}[fragile]{Example}
\protect\hypertarget{example-1}{}
\begin{Shaded}
\begin{Highlighting}[]
\NormalTok{(EXP }\OtherTok{\textless{}{-}}\NormalTok{ CST}\SpecialCharTok{$}\NormalTok{expected)}
\end{Highlighting}
\end{Shaded}

\begin{verbatim}
        survived
sex             0        1
  female 288.0015 177.9985
  male   520.9985 322.0015
\end{verbatim}

\begin{Shaded}
\begin{Highlighting}[]
\NormalTok{(OBS }\OtherTok{\textless{}{-}}\NormalTok{ CST}\SpecialCharTok{$}\NormalTok{observed)}
\end{Highlighting}
\end{Shaded}

\begin{verbatim}
        survived
sex        0   1
  female 127 339
  male   682 161
\end{verbatim}

\begin{Shaded}
\begin{Highlighting}[]
\NormalTok{(chi\_obs }\OtherTok{\textless{}{-}} \FunctionTok{sum}\NormalTok{((OBS }\SpecialCharTok{{-}}\NormalTok{ EXP)}\SpecialCharTok{\^{}}\DecValTok{2}\SpecialCharTok{/}\NormalTok{EXP))}
\end{Highlighting}
\end{Shaded}

\begin{verbatim}
[1] 365.8869
\end{verbatim}
\end{frame}

\hypertarget{chi-square-tests-of-homogeneity}{%
\section{Chi-Square Tests of
Homogeneity}\label{chi-square-tests-of-homogeneity}}

\begin{frame}[fragile]{Example}
\protect\hypertarget{example-2}{}
\begin{itemize}
\tightlist
\item
  Data will often come summarized in contingency tables.
\end{itemize}

\begin{Shaded}
\begin{Highlighting}[]
\NormalTok{DP }\OtherTok{\textless{}{-}} \FunctionTok{c}\NormalTok{(}\DecValTok{67}\NormalTok{, }\DecValTok{76}\NormalTok{, }\DecValTok{57}\NormalTok{, }\DecValTok{48}\NormalTok{, }\DecValTok{73}\NormalTok{, }\DecValTok{79}\NormalTok{)}
\NormalTok{MDP }\OtherTok{\textless{}{-}} \FunctionTok{matrix}\NormalTok{(}\AttributeTok{data =}\NormalTok{ DP, }\AttributeTok{nrow =} \DecValTok{2}\NormalTok{, }\AttributeTok{byrow =} \ConstantTok{TRUE}\NormalTok{)}
\FunctionTok{dimnames}\NormalTok{(MDP) }\OtherTok{\textless{}{-}} \FunctionTok{list}\NormalTok{(}\AttributeTok{Pop =} \FunctionTok{c}\NormalTok{(}\StringTok{"Drug"}\NormalTok{, }\StringTok{"Placebo"}\NormalTok{), }
    \AttributeTok{Status =} \FunctionTok{c}\NormalTok{(}\StringTok{"Improve"}\NormalTok{, }\StringTok{"No Change"}\NormalTok{, }\StringTok{"Worse"}\NormalTok{))}
\NormalTok{TDP }\OtherTok{\textless{}{-}} \FunctionTok{as.table}\NormalTok{(MDP)}
\NormalTok{TDP}
\end{Highlighting}
\end{Shaded}

\begin{verbatim}
         Status
Pop       Improve No Change Worse
  Drug         67        76    57
  Placebo      48        73    79
\end{verbatim}
\end{frame}

\begin{frame}[fragile]{Putting the data back in a tidy format}
\protect\hypertarget{putting-the-data-back-in-a-tidy-format}{}
\begin{Shaded}
\begin{Highlighting}[]
\FunctionTok{library}\NormalTok{(tidyverse)}
\NormalTok{NT }\OtherTok{\textless{}{-}}\NormalTok{ TDP }\SpecialCharTok{\%\textgreater{}\%} 
\NormalTok{  tibble}\SpecialCharTok{::}\FunctionTok{as\_tibble}\NormalTok{() }\SpecialCharTok{\%\textgreater{}\%} 
  \FunctionTok{uncount}\NormalTok{(n)}
\FunctionTok{head}\NormalTok{(NT, }\DecValTok{3}\NormalTok{)}
\end{Highlighting}
\end{Shaded}

\begin{verbatim}
# A tibble: 3 x 2
  Pop   Status 
  <chr> <chr>  
1 Drug  Improve
2 Drug  Improve
3 Drug  Improve
\end{verbatim}
\end{frame}

\end{document}
