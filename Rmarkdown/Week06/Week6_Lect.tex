% Options for packages loaded elsewhere
\PassOptionsToPackage{unicode}{hyperref}
\PassOptionsToPackage{hyphens}{url}
%
\documentclass[
  ignorenonframetext,
]{beamer}
\usepackage{pgfpages}
\setbeamertemplate{caption}[numbered]
\setbeamertemplate{caption label separator}{: }
\setbeamercolor{caption name}{fg=normal text.fg}
\beamertemplatenavigationsymbolsempty
% Prevent slide breaks in the middle of a paragraph
\widowpenalties 1 10000
\raggedbottom
\setbeamertemplate{part page}{
  \centering
  \begin{beamercolorbox}[sep=16pt,center]{part title}
    \usebeamerfont{part title}\insertpart\par
  \end{beamercolorbox}
}
\setbeamertemplate{section page}{
  \centering
  \begin{beamercolorbox}[sep=12pt,center]{part title}
    \usebeamerfont{section title}\insertsection\par
  \end{beamercolorbox}
}
\setbeamertemplate{subsection page}{
  \centering
  \begin{beamercolorbox}[sep=8pt,center]{part title}
    \usebeamerfont{subsection title}\insertsubsection\par
  \end{beamercolorbox}
}
\AtBeginPart{
  \frame{\partpage}
}
\AtBeginSection{
  \ifbibliography
  \else
    \frame{\sectionpage}
  \fi
}
\AtBeginSubsection{
  \frame{\subsectionpage}
}
\usepackage{amsmath,amssymb}
\usepackage{iftex}
\ifPDFTeX
  \usepackage[T1]{fontenc}
  \usepackage[utf8]{inputenc}
  \usepackage{textcomp} % provide euro and other symbols
\else % if luatex or xetex
  \usepackage{unicode-math} % this also loads fontspec
  \defaultfontfeatures{Scale=MatchLowercase}
  \defaultfontfeatures[\rmfamily]{Ligatures=TeX,Scale=1}
\fi
\usepackage{lmodern}
\usetheme[]{Madrid}
\usecolortheme{orchid}
\usefonttheme{professionalfonts}
\ifPDFTeX\else
  % xetex/luatex font selection
\fi
% Use upquote if available, for straight quotes in verbatim environments
\IfFileExists{upquote.sty}{\usepackage{upquote}}{}
\IfFileExists{microtype.sty}{% use microtype if available
  \usepackage[]{microtype}
  \UseMicrotypeSet[protrusion]{basicmath} % disable protrusion for tt fonts
}{}
\makeatletter
\@ifundefined{KOMAClassName}{% if non-KOMA class
  \IfFileExists{parskip.sty}{%
    \usepackage{parskip}
  }{% else
    \setlength{\parindent}{0pt}
    \setlength{\parskip}{6pt plus 2pt minus 1pt}}
}{% if KOMA class
  \KOMAoptions{parskip=half}}
\makeatother
\usepackage{xcolor}
\newif\ifbibliography
\usepackage{color}
\usepackage{fancyvrb}
\newcommand{\VerbBar}{|}
\newcommand{\VERB}{\Verb[commandchars=\\\{\}]}
\DefineVerbatimEnvironment{Highlighting}{Verbatim}{commandchars=\\\{\}}
% Add ',fontsize=\small' for more characters per line
\usepackage{framed}
\definecolor{shadecolor}{RGB}{248,248,248}
\newenvironment{Shaded}{\begin{snugshade}}{\end{snugshade}}
\newcommand{\AlertTok}[1]{\textcolor[rgb]{0.94,0.16,0.16}{#1}}
\newcommand{\AnnotationTok}[1]{\textcolor[rgb]{0.56,0.35,0.01}{\textbf{\textit{#1}}}}
\newcommand{\AttributeTok}[1]{\textcolor[rgb]{0.13,0.29,0.53}{#1}}
\newcommand{\BaseNTok}[1]{\textcolor[rgb]{0.00,0.00,0.81}{#1}}
\newcommand{\BuiltInTok}[1]{#1}
\newcommand{\CharTok}[1]{\textcolor[rgb]{0.31,0.60,0.02}{#1}}
\newcommand{\CommentTok}[1]{\textcolor[rgb]{0.56,0.35,0.01}{\textit{#1}}}
\newcommand{\CommentVarTok}[1]{\textcolor[rgb]{0.56,0.35,0.01}{\textbf{\textit{#1}}}}
\newcommand{\ConstantTok}[1]{\textcolor[rgb]{0.56,0.35,0.01}{#1}}
\newcommand{\ControlFlowTok}[1]{\textcolor[rgb]{0.13,0.29,0.53}{\textbf{#1}}}
\newcommand{\DataTypeTok}[1]{\textcolor[rgb]{0.13,0.29,0.53}{#1}}
\newcommand{\DecValTok}[1]{\textcolor[rgb]{0.00,0.00,0.81}{#1}}
\newcommand{\DocumentationTok}[1]{\textcolor[rgb]{0.56,0.35,0.01}{\textbf{\textit{#1}}}}
\newcommand{\ErrorTok}[1]{\textcolor[rgb]{0.64,0.00,0.00}{\textbf{#1}}}
\newcommand{\ExtensionTok}[1]{#1}
\newcommand{\FloatTok}[1]{\textcolor[rgb]{0.00,0.00,0.81}{#1}}
\newcommand{\FunctionTok}[1]{\textcolor[rgb]{0.13,0.29,0.53}{\textbf{#1}}}
\newcommand{\ImportTok}[1]{#1}
\newcommand{\InformationTok}[1]{\textcolor[rgb]{0.56,0.35,0.01}{\textbf{\textit{#1}}}}
\newcommand{\KeywordTok}[1]{\textcolor[rgb]{0.13,0.29,0.53}{\textbf{#1}}}
\newcommand{\NormalTok}[1]{#1}
\newcommand{\OperatorTok}[1]{\textcolor[rgb]{0.81,0.36,0.00}{\textbf{#1}}}
\newcommand{\OtherTok}[1]{\textcolor[rgb]{0.56,0.35,0.01}{#1}}
\newcommand{\PreprocessorTok}[1]{\textcolor[rgb]{0.56,0.35,0.01}{\textit{#1}}}
\newcommand{\RegionMarkerTok}[1]{#1}
\newcommand{\SpecialCharTok}[1]{\textcolor[rgb]{0.81,0.36,0.00}{\textbf{#1}}}
\newcommand{\SpecialStringTok}[1]{\textcolor[rgb]{0.31,0.60,0.02}{#1}}
\newcommand{\StringTok}[1]{\textcolor[rgb]{0.31,0.60,0.02}{#1}}
\newcommand{\VariableTok}[1]{\textcolor[rgb]{0.00,0.00,0.00}{#1}}
\newcommand{\VerbatimStringTok}[1]{\textcolor[rgb]{0.31,0.60,0.02}{#1}}
\newcommand{\WarningTok}[1]{\textcolor[rgb]{0.56,0.35,0.01}{\textbf{\textit{#1}}}}
\usepackage{longtable,booktabs,array}
\usepackage{calc} % for calculating minipage widths
\usepackage{caption}
% Make caption package work with longtable
\makeatletter
\def\fnum@table{\tablename~\thetable}
\makeatother
\setlength{\emergencystretch}{3em} % prevent overfull lines
\providecommand{\tightlist}{%
  \setlength{\itemsep}{0pt}\setlength{\parskip}{0pt}}
\setcounter{secnumdepth}{-\maxdimen} % remove section numbering
\usepackage{placeins}
\usepackage{color}
\usepackage{bm}
\usepackage{amsmath}
\usepackage{algorithm}
\usepackage[]{algpseudocode}
\usepackage{tabularx}
\usepackage{multirow}
\usepackage[most]{tcolorbox}
\usepackage{tikz}
\usepackage{lipsum}
\usepackage{mathtools}
\usepackage{actuarialangle}
\usepackage{multirow, longtable, array, dcolumn}
\usepackage{tabu}
\newcommand{\sdt}{\bullet}
\newcommand{\tss}{\textsuperscript}
\newcommand{\morearraysp}{\setlength{\arraycolsep}{2mm}}
\newcommand{\smarraysp}{\setlength{\arraycolsep}{1mm}}
\newcommand{\oldarraysp}{\setlength{\arraycolsep}{1.5pt}}
\newcommand{\matrixstretch}{\setlength{\extrarowheight}{4pt}}
\newcommand{\matrixnostretch}{\setlength{\extrarowheight}{0pt}}
\newcommand{\gil}[1]{\textrm{\gilfont{#1}}\normalfont }
\newfont{\gilfont}{msbm10 scaled 1000}
\newcommand{\DOT}{\usebox{\biggercirc}}
\newcommand{\pv}{\wp\text{-value}}
\ifLuaTeX
  \usepackage{selnolig}  % disable illegal ligatures
\fi
\IfFileExists{bookmark.sty}{\usepackage{bookmark}}{\usepackage{hyperref}}
\IfFileExists{xurl.sty}{\usepackage{xurl}}{} % add URL line breaks if available
\urlstyle{same}
\hypersetup{
  pdftitle={STT 3850 : Week 6},
  pdfauthor={Spring 2024},
  hidelinks,
  pdfcreator={LaTeX via pandoc}}

\title{STT 3850 : Week 6}
\author{Spring 2024}
\date{}
\institute{Appalachian State University}

\begin{document}
\frame{\titlepage}

\hypertarget{outline-for-the-week}{%
\section{Outline for the week}\label{outline-for-the-week}}

\begin{frame}{By the end of the week: Multiple Linear Regression}
\protect\hypertarget{by-the-end-of-the-week-multiple-linear-regression}{}
\begin{itemize}
\tightlist
\item
  Extra Sums of Squares
\item
  Model selection
\end{itemize}
\end{frame}

\hypertarget{extra-sums-of-squares}{%
\section{Extra Sums of Squares}\label{extra-sums-of-squares}}

\begin{frame}{Partition of Total sum of squares}
\protect\hypertarget{partition-of-total-sum-of-squares}{}
\begin{itemize}
\item
  For the linear regression model:
  \[y_i=\beta_0+\beta_1x_{i,1}+\ldots + \beta_{p-1}x_{i,p-1}+\epsilon_i\]
\item
  We fit the line:
  \[\hat{y}_i=b_0+b_1x_{i,1}+\ldots + b_{p-1}x_{i,p-1}\]
\item
  Partition of Total sum of squares:

  \begin{itemize}
  \item
    Total sum of squares (\(SST\)): \(SST=\sum(y_i-\bar{y})^2.\)
  \item
    Error sum of squares (\(SSE\)): \(SSE=\sum (y_i-\hat{y}_i)^2.\)
  \item
    Regression sum of squares (\(SSR\)):
    \(SSR=\sum (\hat{y}_i-\bar{y})^2\)
  \end{itemize}
\end{itemize}
\end{frame}

\begin{frame}{Extra Sums of Squares}
\protect\hypertarget{extra-sums-of-squares-1}{}
An extra sum of squares measures the marginal reduction in the error sum
of squares when one or several predictor variables are added to the
regression model, given that the other predictor variables are already
in the model.
\end{frame}

\begin{frame}{Term Life Insurance Example}
\protect\hypertarget{term-life-insurance-example}{}
Like all firms, life insurance companies continually seek new ways to
deliver products to the market. Those involved in product development
want to know who buys insurance and how much they buy. In this example,
we examine the Survey of Consumer Finances (SCF), that contains
extensive information on assets, liabilities, income, and demographic
characteristics of those sampled (potential U.S. customers). We study a
random sample of 500 households with positive incomes that were
interviewed in the 2004 survey.

\textbf{Example: Term Life Insurance}

\begin{itemize}
\tightlist
\item
  \(y\): FACE amount (log scale)
\item
  \(x_1\): Annual Income (log scale)
\item
  \(x_2\): Education
\item
  \(x_3\): Number of household members
\end{itemize}
\end{frame}

\begin{frame}[fragile]{Term Life Insurance Example}
\protect\hypertarget{term-life-insurance-example-1}{}
\normalsize

\begin{Shaded}
\begin{Highlighting}[]
\FunctionTok{library}\NormalTok{(tidyverse)}
\FunctionTok{library}\NormalTok{(moderndive)}
\FunctionTok{library}\NormalTok{(janitor)}
\NormalTok{Term }\OtherTok{\textless{}{-}} \FunctionTok{read.csv}\NormalTok{(}\StringTok{"TermLife.csv"}\NormalTok{)}
\NormalTok{term }\OtherTok{\textless{}{-}}\NormalTok{ Term }\SpecialCharTok{\%\textgreater{}\%} 
  \FunctionTok{clean\_names}\NormalTok{() }\SpecialCharTok{\%\textgreater{}\%} 
  \FunctionTok{filter}\NormalTok{(face }\SpecialCharTok{\textgreater{}} \DecValTok{0}\NormalTok{) }\SpecialCharTok{\%\textgreater{}\%} 
  \FunctionTok{mutate}\NormalTok{(}\AttributeTok{ln\_face =} \FunctionTok{log}\NormalTok{(face), }\AttributeTok{ln\_income =} \FunctionTok{log}\NormalTok{(income)) }\SpecialCharTok{\%\textgreater{}\%} 
  \FunctionTok{select}\NormalTok{(education, ln\_face, ln\_income, numhh)}
\end{Highlighting}
\end{Shaded}

\normalsize
\end{frame}

\begin{frame}[fragile]{Extra Sums of Squares: \(y\) on \(x_1\)}
\protect\hypertarget{extra-sums-of-squares-y-on-x_1}{}
\footnotesize

\begin{Shaded}
\begin{Highlighting}[]
\NormalTok{modelX1 }\OtherTok{\textless{}{-}} \FunctionTok{lm}\NormalTok{(ln\_face }\SpecialCharTok{\textasciitilde{}}\NormalTok{ ln\_income, }\AttributeTok{data =}\NormalTok{ term)}
\FunctionTok{summary}\NormalTok{(modelX1)}
\end{Highlighting}
\end{Shaded}

\begin{verbatim}

Call:
lm(formula = ln_face ~ ln_income, data = term)

Residuals:
    Min      1Q  Median      3Q     Max 
-6.1967 -0.8032 -0.0018  0.8954  6.4711 

Coefficients:
            Estimate Std. Error t value Pr(>|t|)    
(Intercept)  4.23003    0.85985   4.920  1.5e-06 ***
ln_income    0.69604    0.07661   9.086  < 2e-16 ***
---
Signif. codes:  0 '***' 0.001 '**' 0.01 '*' 0.05 '.' 0.1 ' ' 1

Residual standard error: 1.642 on 273 degrees of freedom
Multiple R-squared:  0.2322,    Adjusted R-squared:  0.2294 
F-statistic: 82.55 on 1 and 273 DF,  p-value: < 2.2e-16
\end{verbatim}

\normalsize
\end{frame}

\begin{frame}[fragile]{Extra Sums of Squares: \(y\) on \(x_1\)}
\protect\hypertarget{extra-sums-of-squares-y-on-x_1-1}{}
\normalsize

\begin{Shaded}
\begin{Highlighting}[]
\NormalTok{eis }\OtherTok{\textless{}{-}} \FunctionTok{resid}\NormalTok{(modelX1)}
\NormalTok{SSE }\OtherTok{\textless{}{-}} \FunctionTok{sum}\NormalTok{(eis}\SpecialCharTok{\^{}}\DecValTok{2}\NormalTok{)}
\NormalTok{SST }\OtherTok{\textless{}{-}} \FunctionTok{sum}\NormalTok{((term}\SpecialCharTok{$}\NormalTok{ln\_face }\SpecialCharTok{{-}} \FunctionTok{mean}\NormalTok{(term}\SpecialCharTok{$}\NormalTok{ln\_face))}\SpecialCharTok{\^{}}\DecValTok{2}\NormalTok{)}
\NormalTok{SSR }\OtherTok{\textless{}{-}}\NormalTok{ SST }\SpecialCharTok{{-}}\NormalTok{ SSE}
\FunctionTok{c}\NormalTok{(SSE, SSR)}
\end{Highlighting}
\end{Shaded}

\begin{verbatim}
[1] 736.2671 222.6292
\end{verbatim}

\begin{Shaded}
\begin{Highlighting}[]
\NormalTok{knitr}\SpecialCharTok{::}\FunctionTok{kable}\NormalTok{(}\FunctionTok{anova}\NormalTok{(modelX1))}
\end{Highlighting}
\end{Shaded}

\begin{longtable}[]{@{}lrrrrr@{}}
\toprule\noalign{}
& Df & Sum Sq & Mean Sq & F value & Pr(\textgreater F) \\
\midrule\noalign{}
\endhead
ln\_income & 1 & 222.6292 & 222.629245 & 82.54855 & 0 \\
Residuals & 273 & 736.2671 & 2.696949 & NA & NA \\
\bottomrule\noalign{}
\end{longtable}

\normalsize
\end{frame}

\begin{frame}[fragile]{Extra Sums of Squares: \(y\) on \(x_2\)}
\protect\hypertarget{extra-sums-of-squares-y-on-x_2}{}
\footnotesize

\begin{Shaded}
\begin{Highlighting}[]
\NormalTok{modelX2 }\OtherTok{\textless{}{-}} \FunctionTok{lm}\NormalTok{(ln\_face }\SpecialCharTok{\textasciitilde{}}\NormalTok{ education, }\AttributeTok{data =}\NormalTok{ term)}
\FunctionTok{summary}\NormalTok{(modelX2)}
\end{Highlighting}
\end{Shaded}

\begin{verbatim}

Call:
lm(formula = ln_face ~ education, data = term)

Residuals:
    Min      1Q  Median      3Q     Max 
-5.4395 -1.2698  0.2065  1.2194  4.5559 

Coefficients:
            Estimate Std. Error t value Pr(>|t|)    
(Intercept)  7.90986    0.60499  13.074  < 2e-16 ***
education    0.28095    0.04103   6.847 4.96e-11 ***
---
Signif. codes:  0 '***' 0.001 '**' 0.01 '*' 0.05 '.' 0.1 ' ' 1

Residual standard error: 1.731 on 273 degrees of freedom
Multiple R-squared:  0.1466,    Adjusted R-squared:  0.1434 
F-statistic: 46.89 on 1 and 273 DF,  p-value: 4.964e-11
\end{verbatim}

\normalsize
\end{frame}

\begin{frame}[fragile]{Extra Sums of Squares: \(y\) on \(x_2\)}
\protect\hypertarget{extra-sums-of-squares-y-on-x_2-1}{}
\normalsize

\begin{Shaded}
\begin{Highlighting}[]
\FunctionTok{get\_regression\_points}\NormalTok{(modelX2) }\OtherTok{{-}\textgreater{}}\NormalTok{ RT}
\NormalTok{RT }\SpecialCharTok{\%\textgreater{}\%} 
  \FunctionTok{summarize}\NormalTok{(}\AttributeTok{SSE =} \FunctionTok{sum}\NormalTok{(residual}\SpecialCharTok{\^{}}\DecValTok{2}\NormalTok{), }
            \AttributeTok{SST =} \FunctionTok{sum}\NormalTok{((ln\_face }\SpecialCharTok{{-}} \FunctionTok{mean}\NormalTok{(ln\_face))}\SpecialCharTok{\^{}}\DecValTok{2}\NormalTok{),}
            \AttributeTok{SSR =}\NormalTok{ SST }\SpecialCharTok{{-}}\NormalTok{ SSE) }\OtherTok{{-}\textgreater{}}\NormalTok{ ESS}
\NormalTok{knitr}\SpecialCharTok{::}\FunctionTok{kable}\NormalTok{(ESS)}
\end{Highlighting}
\end{Shaded}

\begin{longtable}[]{@{}rrr@{}}
\toprule\noalign{}
SSE & SST & SSR \\
\midrule\noalign{}
\endhead
818.375 & 958.967 & 140.5921 \\
\bottomrule\noalign{}
\end{longtable}

\begin{Shaded}
\begin{Highlighting}[]
\NormalTok{knitr}\SpecialCharTok{::}\FunctionTok{kable}\NormalTok{(}\FunctionTok{anova}\NormalTok{(modelX2))}
\end{Highlighting}
\end{Shaded}

\begin{longtable}[]{@{}lrrrrr@{}}
\toprule\noalign{}
& Df & Sum Sq & Mean Sq & F value & Pr(\textgreater F) \\
\midrule\noalign{}
\endhead
education & 1 & 140.5486 & 140.548609 & 46.88688 & 0 \\
Residuals & 273 & 818.3477 & 2.997611 & NA & NA \\
\bottomrule\noalign{}
\end{longtable}

\normalsize
\end{frame}

\begin{frame}[fragile]{Extra Sums of Squares: \(y\) on \(x_1\) and
\(x_2\)}
\protect\hypertarget{extra-sums-of-squares-y-on-x_1-and-x_2}{}
\footnotesize

\begin{Shaded}
\begin{Highlighting}[]
\NormalTok{modelX1X2 }\OtherTok{\textless{}{-}} \FunctionTok{lm}\NormalTok{(ln\_face }\SpecialCharTok{\textasciitilde{}}\NormalTok{ ln\_income }\SpecialCharTok{+}\NormalTok{ education, }\AttributeTok{data =}\NormalTok{ term)}
\FunctionTok{summary}\NormalTok{(modelX1X2)}
\end{Highlighting}
\end{Shaded}

\begin{verbatim}

Call:
lm(formula = ln_face ~ ln_income + education, data = term)

Residuals:
    Min      1Q  Median      3Q     Max 
-6.1266 -1.0284  0.1817  0.9185  5.3403 

Coefficients:
            Estimate Std. Error t value Pr(>|t|)    
(Intercept)  2.96235    0.87676   3.379 0.000835 ***
ln_income    0.57392    0.07879   7.284 3.50e-12 ***
education    0.18103    0.04003   4.523 9.11e-06 ***
---
Signif. codes:  0 '***' 0.001 '**' 0.01 '*' 0.05 '.' 0.1 ' ' 1

Residual standard error: 1.587 on 272 degrees of freedom
Multiple R-squared:  0.2859,    Adjusted R-squared:  0.2806 
F-statistic: 54.44 on 2 and 272 DF,  p-value: < 2.2e-16
\end{verbatim}

\normalsize
\end{frame}

\begin{frame}[fragile]{Extra Sums of Squares: \(y\) on \(x_1\) and
\(x_2\)}
\protect\hypertarget{extra-sums-of-squares-y-on-x_1-and-x_2-1}{}
\normalsize

\begin{Shaded}
\begin{Highlighting}[]
\FunctionTok{get\_regression\_points}\NormalTok{(modelX1X2) }\OtherTok{{-}\textgreater{}}\NormalTok{ RT2}
\NormalTok{RT2 }\SpecialCharTok{\%\textgreater{}\%} 
  \FunctionTok{summarize}\NormalTok{(}\AttributeTok{SSE =} \FunctionTok{sum}\NormalTok{(residual}\SpecialCharTok{\^{}}\DecValTok{2}\NormalTok{), }
            \AttributeTok{SST =} \FunctionTok{sum}\NormalTok{((ln\_face }\SpecialCharTok{{-}} \FunctionTok{mean}\NormalTok{(ln\_face))}\SpecialCharTok{\^{}}\DecValTok{2}\NormalTok{),}
            \AttributeTok{SSR =}\NormalTok{ SST }\SpecialCharTok{{-}}\NormalTok{ SSE) }\OtherTok{{-}\textgreater{}}\NormalTok{ ESS2}
\NormalTok{knitr}\SpecialCharTok{::}\FunctionTok{kable}\NormalTok{(ESS2)}
\end{Highlighting}
\end{Shaded}

\begin{longtable}[]{@{}rrr@{}}
\toprule\noalign{}
SSE & SST & SSR \\
\midrule\noalign{}
\endhead
684.7941 & 958.967 & 274.1729 \\
\bottomrule\noalign{}
\end{longtable}

\begin{Shaded}
\begin{Highlighting}[]
\NormalTok{knitr}\SpecialCharTok{::}\FunctionTok{kable}\NormalTok{(}\FunctionTok{anova}\NormalTok{(modelX1X2))}
\end{Highlighting}
\end{Shaded}

\begin{longtable}[]{@{}lrrrrr@{}}
\toprule\noalign{}
& Df & Sum Sq & Mean Sq & F value & Pr(\textgreater F) \\
\midrule\noalign{}
\endhead
ln\_income & 1 & 222.6292 & 222.629245 & 88.43204 & 0.0e+00 \\
education & 1 & 51.5022 & 51.502201 & 20.45753 & 9.1e-06 \\
Residuals & 272 & 684.7649 & 2.517518 & NA & NA \\
\bottomrule\noalign{}
\end{longtable}

\normalsize
\end{frame}

\begin{frame}[fragile]{Extra Sums of Squares: \(y\) on \(x_1\), \(x_2\)
and \(x_3\)}
\protect\hypertarget{extra-sums-of-squares-y-on-x_1-x_2-and-x_3}{}
\scriptsize

\begin{Shaded}
\begin{Highlighting}[]
\NormalTok{modelAll }\OtherTok{\textless{}{-}} \FunctionTok{lm}\NormalTok{(ln\_face }\SpecialCharTok{\textasciitilde{}}\NormalTok{ ln\_income }\SpecialCharTok{+}\NormalTok{ education }\SpecialCharTok{+}\NormalTok{ numhh, }\AttributeTok{data =}\NormalTok{ term)}
\FunctionTok{summary}\NormalTok{(modelAll)}
\end{Highlighting}
\end{Shaded}

\begin{verbatim}

Call:
lm(formula = ln_face ~ ln_income + education + numhh, data = term)

Residuals:
    Min      1Q  Median      3Q     Max 
-5.7420 -0.8681  0.0549  0.9093  4.7187 

Coefficients:
            Estimate Std. Error t value Pr(>|t|)    
(Intercept)  2.58408    0.84643   3.053  0.00249 ** 
ln_income    0.49353    0.07754   6.365 8.32e-10 ***
education    0.20641    0.03883   5.316 2.22e-07 ***
numhh        0.30605    0.06333   4.833 2.26e-06 ***
---
Signif. codes:  0 '***' 0.001 '**' 0.01 '*' 0.05 '.' 0.1 ' ' 1

Residual standard error: 1.525 on 271 degrees of freedom
Multiple R-squared:  0.3425,    Adjusted R-squared:  0.3353 
F-statistic: 47.07 on 3 and 271 DF,  p-value: < 2.2e-16
\end{verbatim}

\normalsize
\end{frame}

\begin{frame}[fragile]{Extra Sums of Squares: \(y\) on \(x_1\), \(x_2\)
and \(x_3\)}
\protect\hypertarget{extra-sums-of-squares-y-on-x_1-x_2-and-x_3-1}{}
\normalsize

\begin{Shaded}
\begin{Highlighting}[]
\FunctionTok{get\_regression\_points}\NormalTok{(modelAll) }\OtherTok{{-}\textgreater{}}\NormalTok{ RT3}
\NormalTok{RT3 }\SpecialCharTok{\%\textgreater{}\%} 
  \FunctionTok{summarize}\NormalTok{(}\AttributeTok{SSE =} \FunctionTok{sum}\NormalTok{(residual}\SpecialCharTok{\^{}}\DecValTok{2}\NormalTok{), }
            \AttributeTok{SST =} \FunctionTok{sum}\NormalTok{((ln\_face }\SpecialCharTok{{-}} \FunctionTok{mean}\NormalTok{(ln\_face))}\SpecialCharTok{\^{}}\DecValTok{2}\NormalTok{),}
            \AttributeTok{SSR =}\NormalTok{ SST }\SpecialCharTok{{-}}\NormalTok{ SSE) }\OtherTok{{-}\textgreater{}}\NormalTok{ ESS3}
\NormalTok{knitr}\SpecialCharTok{::}\FunctionTok{kable}\NormalTok{(ESS3)}
\end{Highlighting}
\end{Shaded}

\begin{longtable}[]{@{}rrr@{}}
\toprule\noalign{}
SSE & SST & SSR \\
\midrule\noalign{}
\endhead
630.458 & 958.967 & 328.509 \\
\bottomrule\noalign{}
\end{longtable}

\begin{Shaded}
\begin{Highlighting}[]
\NormalTok{knitr}\SpecialCharTok{::}\FunctionTok{kable}\NormalTok{(}\FunctionTok{anova}\NormalTok{(modelAll))}
\end{Highlighting}
\end{Shaded}

\begin{longtable}[]{@{}lrrrrr@{}}
\toprule\noalign{}
& Df & Sum Sq & Mean Sq & F value & Pr(\textgreater F) \\
\midrule\noalign{}
\endhead
ln\_income & 1 & 222.62925 & 222.629245 & 95.70075 & 0.0e+00 \\
education & 1 & 51.50220 & 51.502201 & 22.13905 & 4.0e-06 \\
numhh & 1 & 54.33593 & 54.335932 & 23.35717 & 2.3e-06 \\
Residuals & 271 & 630.42897 & 2.326306 & NA & NA \\
\bottomrule\noalign{}
\end{longtable}

\normalsize
\end{frame}

\begin{frame}{Extra Sums of Squares}
\protect\hypertarget{extra-sums-of-squares-2}{}
\begin{table}[h]
\centering
\resizebox{0.7\textheight}{!}{
    \medskip{}\begin{tabular}{lll}
\hline
Independent Variable      &  SSR  &  SSE  \\
\hline
$x_1$ & 222.63 & 736.27 \\
$x_2$ & 140.55 & 818.35 \\
$x_1$ and $x_2$ & 274.13 & 684.76 \\
$x_1$, $x_2$ and $x_3$  & 328.47 & 630.43 \\

\hline
        \end{tabular}
    }
\end{table}

Hence: \[SSR(x_2|x_1)=SSE(x_1)-SSE(x_1, x_2)=736.27-684.76=51.51\] or
\[SSR(x_2|x_1)=SSR(x_1,x_2)-SSR(x_1)=274.13-222.63=51.51\]

Question:

\begin{enumerate}
\tightlist
\item
  Why are they equal?
\item
  Find \(SSR(x_1|x_2)\)?
\end{enumerate}
\end{frame}

\begin{frame}{Extra Sums of Squares}
\protect\hypertarget{extra-sums-of-squares-3}{}
\begin{table}[h]
\centering
\resizebox{0.7\textheight}{!}{
    \medskip{}\begin{tabular}{lll}
\hline
Independent Variable      &  SSR  &  SSE  \\
\hline
$x_1$ & 222.63 & 736.27 \\
$x_2$ & 140.55 & 818.35 \\
$x_1$ and $x_2$ & 274.13 & 684.76 \\
$x_1$, $x_2$ and $x_3$  & 328.47 & 630.43 \\

\hline
        \end{tabular}
    }
\end{table}

Similarly:
\[\begin{array}{ll}SSR(x_3|x_1,x_2)&=SSE(x_1,x_2)-SSE(x_1, x_2, x_3)\\
&=684.76-630.43=54.33\end{array}\] or
\[\begin{array}{ll}SSR(x_3|x_1,x_2)&=SSR(x_1,x_2,x_3)-SSR(x_1,x_2)\\
&=328.47-274.13=54.33\end{array}\]

Problem: Find the value of \(SSR(x_2,x_3|x_1)\).
\end{frame}

\begin{frame}{Decomposition}
\protect\hypertarget{decomposition}{}
In multiple regression, we can obtain a variety of decompositions of the
regression \(SSR\) into extra sum of squares. For example,

\[SSR(x_1, x_2)=SSR(x_1)+SSR(x_2|x_1), \quad \text{or}\]
\[SSR(x_1, x_2)=SSR(x_2)+SSR(x_1|x_2).\] If we have three variables,
then:
\[SSR(x_1, x_2,x_3)=SSR(x_1)+SSR(x_2|x_1)+SSR(x_3|x_1,x_2), \quad \text{or}\]
\[SSR(x_1, x_2,x_3)=SSR(x_2)+SSR(x_3|x_2)+SSR(x_1|x_2,x_3), \quad \text{or}\]
\[SSR(x_1, x_2,x_3)=SSR(x_1)+SSR(x_2,x_3|x_1).\]
\end{frame}

\begin{frame}{Analysis of Variance: ANOVA}
\protect\hypertarget{analysis-of-variance-anova}{}
The ANOVA table is shown below

\begin{table}[h]
\centering
\resizebox{1.2\textheight}{!}{
    \medskip{}\begin{tabular}{lllll}
\hline
     Source of       &  Degrees of  &    &               &                  \\
     Variation       &  Freedom      &  Sum of Squares   &  Mean Square &   \\
     (Source)        &   ($df$)      &  ($SS$)             & ($MS$)         &  $F$ \\
\hline\hline
$x_1$ & $1$ & $SSR(x_1)$ & $MSR(x_1)=SSR(x_1)/1$ & $F=\frac{MSR((x_1))}{MSE((x_1,x_2,x_3))}$\\
$x_2|x_1$ & $1$ & $SSR(x_2|x_1)$ & $MSR(x_2|x_1)=SSR(x_2|x_1)/1$ & $F=\frac{MSR((x_2|x_1))}{MSE((x_1,x_2,x_3))}$\\
$x_3|x_1,x_2$ & $1$ & $SSR(x_3|x_1,x_2)$ & $MSR(x_3|x_1,x_2)=SSR(x_3|x_1,x_2)/1$ & $F=\frac{MSR((x_3|x_1,x_2))}{MSE((x_1,x_2,x_3))}$\\
Error & $n-4$ & $SSE(x_1,x_2,x_3)$ & $MSE=SSE(x_1,x_2,x_3)/(n-4)$ & \\
\hline
Total & $n-1$ & $SST(x_1,x_2,x_3)$ &  & \\

\hline
        \end{tabular}
    }
\end{table}
\end{frame}

\begin{frame}[fragile]{Analysis of Variance: ANOVA}
\protect\hypertarget{analysis-of-variance-anova-1}{}
\normalsize

\begin{Shaded}
\begin{Highlighting}[]
\CommentTok{\#For Term Life Insurance Example:}
\NormalTok{knitr}\SpecialCharTok{::}\FunctionTok{kable}\NormalTok{(}\FunctionTok{anova}\NormalTok{(modelAll))}
\end{Highlighting}
\end{Shaded}

\begin{longtable}[]{@{}lrrrrr@{}}
\toprule\noalign{}
& Df & Sum Sq & Mean Sq & F value & Pr(\textgreater F) \\
\midrule\noalign{}
\endhead
ln\_income & 1 & 222.62925 & 222.629245 & 95.70075 & 0.0e+00 \\
education & 1 & 51.50220 & 51.502201 & 22.13905 & 4.0e-06 \\
numhh & 1 & 54.33593 & 54.335932 & 23.35717 & 2.3e-06 \\
Residuals & 271 & 630.42897 & 2.326306 & NA & NA \\
\bottomrule\noalign{}
\end{longtable}

\normalsize

Why are extra sum of squares of interest?
\end{frame}

\begin{frame}{Tests for Regression Coefficients}
\protect\hypertarget{tests-for-regression-coefficients}{}
When we wish to test whether the term \(\beta_kx_k\) can be dropped from
a multiple regression model, we are interested in:
\[H_0:\beta_k=0, \quad vs. \quad H_a: \beta_k \neq 0.\]
\end{frame}

\begin{frame}{Tests for Regression Coefficients}
\protect\hypertarget{tests-for-regression-coefficients-1}{}
For example, let us consider the first-order regression model
\[y_i=\beta_0+\beta_1 x_{i1}+\beta_2 x_{i2}+\beta_3 x_{i3}+\epsilon_i.\]
Test: \[H_0:\beta_3=0, \quad vs. \quad H_a: \beta_3\neq 0.\] Under the
null, we have the reduced model,
\[y_i=\beta_0+\beta_1 x_{i1}+\beta_2 x_{i2}+\epsilon_i.\] For those two
models, the extra sum of squares is
\[SSR(x_3|x_1,x_2)=SSE(x_1,x_2)-SSE(x_1,x_2,x_3)\]
\end{frame}

\begin{frame}{Tests for Regression Coefficients}
\protect\hypertarget{tests-for-regression-coefficients-2}{}
The general linear test statistic
\[F^*=\frac{SSE_{reduced}-SSE_{full}}{df_{reduced}-df_{full}}\div \frac{SSE_{full}}{df_{full}}\]
becomes:

\[  \begin{array}{ll}
F^* &= \frac{SSE(x_1,x_2)-SSE(x_1,x_2,x_3)}{(n-3)-(n-4)}\div \frac{SSE(x_1,x_2,x_3)}{n-4}\\ 
&=\frac{SSR(x_3|x_1,x_2)}{1}\div \frac{SSE(x_1,x_2,x_3)}{n-4} \\
&=\frac{MSR(x_3|x_1,x_2)}{MSE(x_1,x_2,x_3)}
\end{array}\]
\end{frame}

\begin{frame}[fragile]{Term Life Insurance Example}
\protect\hypertarget{term-life-insurance-example-2}{}
\[F^*=\frac{54.34}{1}\div \frac{630.43}{271}=23.36\] \normalsize

\begin{Shaded}
\begin{Highlighting}[]
\NormalTok{Fstar }\OtherTok{\textless{}{-}} \FunctionTok{anova}\NormalTok{(modelAll)[}\DecValTok{3}\NormalTok{, }\DecValTok{4}\NormalTok{]}
\NormalTok{Fstar}
\end{Highlighting}
\end{Shaded}

\begin{verbatim}
[1] 23.35717
\end{verbatim}

\begin{Shaded}
\begin{Highlighting}[]
\DocumentationTok{\#\# Get p{-}value for F{-}statistic}
\NormalTok{pvalue }\OtherTok{\textless{}{-}} \DecValTok{1} \SpecialCharTok{{-}} \FunctionTok{pf}\NormalTok{(}\FloatTok{23.36}\NormalTok{, }\DecValTok{1}\NormalTok{, }\DecValTok{271}\NormalTok{)}
\NormalTok{pvalue}
\end{Highlighting}
\end{Shaded}

\begin{verbatim}
[1] 2.252657e-06
\end{verbatim}

\normalsize
\end{frame}

\begin{frame}[fragile]{Term Life Insurance Example}
\protect\hypertarget{term-life-insurance-example-3}{}
We can use the \(\wp\text{-value}\) to test \(H_0:\beta_3=0\).

\scriptsize

\begin{Shaded}
\begin{Highlighting}[]
\FunctionTok{summary}\NormalTok{(modelAll)}
\end{Highlighting}
\end{Shaded}

\begin{verbatim}

Call:
lm(formula = ln_face ~ ln_income + education + numhh, data = term)

Residuals:
    Min      1Q  Median      3Q     Max 
-5.7420 -0.8681  0.0549  0.9093  4.7187 

Coefficients:
            Estimate Std. Error t value Pr(>|t|)    
(Intercept)  2.58408    0.84643   3.053  0.00249 ** 
ln_income    0.49353    0.07754   6.365 8.32e-10 ***
education    0.20641    0.03883   5.316 2.22e-07 ***
numhh        0.30605    0.06333   4.833 2.26e-06 ***
---
Signif. codes:  0 '***' 0.001 '**' 0.01 '*' 0.05 '.' 0.1 ' ' 1

Residual standard error: 1.525 on 271 degrees of freedom
Multiple R-squared:  0.3425,    Adjusted R-squared:  0.3353 
F-statistic: 47.07 on 3 and 271 DF,  p-value: < 2.2e-16
\end{verbatim}

\normalsize
\end{frame}

\begin{frame}[fragile]{Testing More Than One Coefficient}
\protect\hypertarget{testing-more-than-one-coefficient}{}
Consider testing:
\[H_0:\beta_2=\beta_3=0, \text{versus} \quad H_a: \text{at least one } \beta_i \neq 0 \text{ for } i = 1, 2, \ldots, p-1.\]

Under the null, we have the reduced model,
\[y_i=\beta_0+\beta_1 x_{i1}+\epsilon_i.\] This is the \texttt{modelX1}
we estimated earlier.
\end{frame}

\begin{frame}[fragile]{Testing More Than One Coefficient}
\protect\hypertarget{testing-more-than-one-coefficient-1}{}
\normalsize

\begin{Shaded}
\begin{Highlighting}[]
\CommentTok{\#For Term Life Insurance Example:}
\FunctionTok{anova}\NormalTok{(modelX1, modelAll)}
\end{Highlighting}
\end{Shaded}

\begin{verbatim}
Analysis of Variance Table

Model 1: ln_face ~ ln_income
Model 2: ln_face ~ ln_income + education + numhh
  Res.Df    RSS Df Sum of Sq      F    Pr(>F)    
1    273 736.27                                  
2    271 630.43  2    105.84 22.748 7.369e-10 ***
---
Signif. codes:  0 '***' 0.001 '**' 0.01 '*' 0.05 '.' 0.1 ' ' 1
\end{verbatim}

\normalsize
\end{frame}

\begin{frame}{Is There a Relationship Between the Response and
Predictors}
\protect\hypertarget{is-there-a-relationship-between-the-response-and-predictors}{}
Now, we test:
\[H_0:\beta_1=\beta_2=\beta_3=0, \text{ versus } H_a: \text{at least one } \beta_i \neq 0 \text{ for } i = 1, 2, \ldots, p-1.\]
Under the null, we have the reduced model, \[y_i=\beta_0+\epsilon_i.\]
\end{frame}

\begin{frame}[fragile]{Is There a Relationship Between the Response and
Predictors}
\protect\hypertarget{is-there-a-relationship-between-the-response-and-predictors-1}{}
\normalsize

\begin{Shaded}
\begin{Highlighting}[]
\NormalTok{modelInt }\OtherTok{\textless{}{-}} \FunctionTok{lm}\NormalTok{(ln\_face }\SpecialCharTok{\textasciitilde{}} \DecValTok{1}\NormalTok{, }\AttributeTok{data =}\NormalTok{ term)}
\FunctionTok{summary}\NormalTok{(modelInt)}
\end{Highlighting}
\end{Shaded}

\begin{verbatim}

Call:
lm(formula = ln_face ~ 1, data = term)

Residuals:
    Min      1Q  Median      3Q     Max 
-5.3057 -1.1705 -0.0719  1.2974  4.4643 

Coefficients:
            Estimate Std. Error t value Pr(>|t|)    
(Intercept)  11.9903     0.1128   106.3   <2e-16 ***
---
Signif. codes:  0 '***' 0.001 '**' 0.01 '*' 0.05 '.' 0.1 ' ' 1

Residual standard error: 1.871 on 274 degrees of freedom
\end{verbatim}

\normalsize
\end{frame}

\begin{frame}[fragile]{Is There a Relationship Between the Response and
Predictors}
\protect\hypertarget{is-there-a-relationship-between-the-response-and-predictors-2}{}
\normalsize

\begin{Shaded}
\begin{Highlighting}[]
\FunctionTok{anova}\NormalTok{(modelInt, modelAll)}
\end{Highlighting}
\end{Shaded}

\begin{verbatim}
Analysis of Variance Table

Model 1: ln_face ~ 1
Model 2: ln_face ~ ln_income + education + numhh
  Res.Df    RSS Df Sum of Sq      F    Pr(>F)    
1    274 958.90                                  
2    271 630.43  3    328.47 47.066 < 2.2e-16 ***
---
Signif. codes:  0 '***' 0.001 '**' 0.01 '*' 0.05 '.' 0.1 ' ' 1
\end{verbatim}

\normalsize
\end{frame}

\begin{frame}[fragile]{Is There a Relationship Between the Response and
Predictors}
\protect\hypertarget{is-there-a-relationship-between-the-response-and-predictors-3}{}
\scriptsize

\begin{Shaded}
\begin{Highlighting}[]
\CommentTok{\# Or}
\FunctionTok{summary}\NormalTok{(modelAll)}
\end{Highlighting}
\end{Shaded}

\begin{verbatim}

Call:
lm(formula = ln_face ~ ln_income + education + numhh, data = term)

Residuals:
    Min      1Q  Median      3Q     Max 
-5.7420 -0.8681  0.0549  0.9093  4.7187 

Coefficients:
            Estimate Std. Error t value Pr(>|t|)    
(Intercept)  2.58408    0.84643   3.053  0.00249 ** 
ln_income    0.49353    0.07754   6.365 8.32e-10 ***
education    0.20641    0.03883   5.316 2.22e-07 ***
numhh        0.30605    0.06333   4.833 2.26e-06 ***
---
Signif. codes:  0 '***' 0.001 '**' 0.01 '*' 0.05 '.' 0.1 ' ' 1

Residual standard error: 1.525 on 271 degrees of freedom
Multiple R-squared:  0.3425,    Adjusted R-squared:  0.3353 
F-statistic: 47.07 on 3 and 271 DF,  p-value: < 2.2e-16
\end{verbatim}

\normalsize
\end{frame}

\hypertarget{model-selection}{%
\section{Model selection}\label{model-selection}}

\begin{frame}{Model selection}
\protect\hypertarget{model-selection-1}{}
The general multiple linear regression model with response \(y\) and
terms \(x_1, \ldots, x_{p-1}\) will have the form:
\[y=\beta_0+\beta_1x_1+\ldots + \beta_{p-1}x_{p-1}+\epsilon.\]

\begin{itemize}
\item
  How many alternative models:

  \begin{itemize}
  \tightlist
  \item
    \(y=\beta_0+\epsilon\)
  \item
    \(y=\beta_0+\beta_1x_1+\epsilon\)
  \item
    \(y=\beta_0+\beta_2x_2+\epsilon\)
  \item
    \(\vdots\)
  \item
    \(y=\beta_0+\beta_{p-1}x_{p-1}+\epsilon\)
  \item
    \(y=\beta_0+\beta_1x_1+\beta_2x_2+\epsilon\)
  \item
    \(\vdots\)
  \item
    \(y=\beta_0+\beta_1x_1+\beta_2x_2+\ldots+\beta_{p-1}x_{p-1}\epsilon\)
  \end{itemize}
\end{itemize}

One can construct a total of \(2^{p-1}\) models! Question: How to select
the ``best'' model?
\end{frame}

\begin{frame}{Partition of Total sum of squares}
\protect\hypertarget{partition-of-total-sum-of-squares-1}{}
\begin{itemize}
\item
  For the linear regression model:
  \[y_i=\beta_0+\beta_1x_{i,1}+\ldots + \beta_{p-1}x_{i,p-1}+\epsilon_i.\]
\item
  We have fitted the line:
  \(\hat{y}_i=b_0+b_1x_{i,1}+\ldots + b_{p-1}x_{i,p-1}.\)
\item
  Partitioning the sum of squares is the same:

  \begin{itemize}
  \item
    Total sum of squares (SST): \(SST=\sum(y_i-\bar{y})^2.\)
  \item
    Error sum of squares (SSE): \(SSE=\sum (y_i-\hat{y}_i)^2.\)
  \item
    Regression sum of squares (SSR): \(SSR=\sum (\hat{y}_i-\bar{y})^2\)
  \end{itemize}
\end{itemize}
\end{frame}

\begin{frame}{\(R^2\) and Adjusted \(R^2\) (\(R^2_{\text{adj}}\))}
\protect\hypertarget{r2-and-adjusted-r2-r2_textadj}{}
The coefficient of determination of the regression model, is defined as
the proportion of the total sample variability in the \(y\)'s explained
by the regression model, that is,
\[R^2=\frac{SSR}{SST}=1-\frac{SSE}{SST}.\] We can also write:

\[R^2=\frac{var(\hat{y})}{var(y)}\]
\end{frame}

\begin{frame}{\(R^2\) and Adjusted \(R^2\) (\(R^2_{\text{adj}}\))}
\protect\hypertarget{r2-and-adjusted-r2-r2_textadj-1}{}
Caution: adding irrelevant predictor variables to the regression
equation often increases \(R^2\).

Q: Why do we use it?

A: The intent in using \(R^2\) criterion is to find the point where
adding more \(x\) variables is not worthwhile because it leads to a very
small increase in \(R^2\). Often this point is reached when only a
limited number of \(x\) variables are included in the regression model.
\end{frame}

\begin{frame}{\(R^2\) and Adjusted \(R^2\) (\(R^2_{\text{adj}}\))}
\protect\hypertarget{r2-and-adjusted-r2-r2_textadj-2}{}
One can define an adjusted coefficient of determination
\[R^2_{\text{adj}}=1-\frac{SSE/(n-p)}{SST/ (n-1)}=1-\frac{MSE}{SST/n-1}\]

where \(p\) is the number of predictors in the current model. This
coefficient takes the number of parameters in the regression model into
account using degrees of freedom.

Users of the \(R^2_{\text{adj}}\) criterion seek to find a few subsets
for which \(R^2_{\text{adj}}\) is at the maximum or that adding more
variables is not worthwhile.
\end{frame}

\begin{frame}[fragile]{Needed packages}
\protect\hypertarget{needed-packages}{}
Let's load all the packages needed for this chapter.

\normalsize

\begin{Shaded}
\begin{Highlighting}[]
\FunctionTok{library}\NormalTok{(tidyverse) }
\FunctionTok{library}\NormalTok{(moderndive)}
\FunctionTok{library}\NormalTok{(skimr)}
\FunctionTok{library}\NormalTok{(ISLR)}
\NormalTok{evals\_ch6 }\OtherTok{\textless{}{-}}\NormalTok{ evals }\SpecialCharTok{\%\textgreater{}\%}
  \FunctionTok{select}\NormalTok{(ID, score, age, gender)}
\end{Highlighting}
\end{Shaded}

\normalsize
\end{frame}

\begin{frame}[fragile]{Compare Interaction and Parallel slopes model}
\protect\hypertarget{compare-interaction-and-parallel-slopes-model}{}
\scriptsize

\begin{Shaded}
\begin{Highlighting}[]
\CommentTok{\# Fit interaction model:}
\NormalTok{score\_model\_interaction }\OtherTok{\textless{}{-}} \FunctionTok{lm}\NormalTok{(score }\SpecialCharTok{\textasciitilde{}}\NormalTok{ age }\SpecialCharTok{+}\NormalTok{ gender }\SpecialCharTok{+}\NormalTok{ age}\SpecialCharTok{:}\NormalTok{gender, }\AttributeTok{data =}\NormalTok{ evals\_ch6)}
\FunctionTok{summary}\NormalTok{(score\_model\_interaction)}
\end{Highlighting}
\end{Shaded}

\begin{verbatim}

Call:
lm(formula = score ~ age + gender + age:gender, data = evals_ch6)

Residuals:
     Min       1Q   Median       3Q      Max 
-1.86453 -0.34815  0.09863  0.40661  0.96327 

Coefficients:
                Estimate Std. Error t value Pr(>|t|)    
(Intercept)     4.882989   0.205210  23.795  < 2e-16 ***
age            -0.017523   0.004472  -3.919 0.000103 ***
gendermale     -0.446044   0.265407  -1.681 0.093520 .  
age:gendermale  0.013531   0.005531   2.446 0.014803 *  
---
Signif. codes:  0 '***' 0.001 '**' 0.01 '*' 0.05 '.' 0.1 ' ' 1

Residual standard error: 0.5314 on 459 degrees of freedom
Multiple R-squared:  0.05138,   Adjusted R-squared:  0.04518 
F-statistic: 8.288 on 3 and 459 DF,  p-value: 2.227e-05
\end{verbatim}

\normalsize
\end{frame}

\begin{frame}[fragile]{Compare Interaction and Parallel slopes model}
\protect\hypertarget{compare-interaction-and-parallel-slopes-model-1}{}
\scriptsize

\begin{Shaded}
\begin{Highlighting}[]
\CommentTok{\# Fit parallel slopes model:}
\NormalTok{score\_model\_parallel\_slopes }\OtherTok{\textless{}{-}} \FunctionTok{lm}\NormalTok{(score }\SpecialCharTok{\textasciitilde{}}\NormalTok{ age }\SpecialCharTok{+}\NormalTok{ gender, }\AttributeTok{data =}\NormalTok{ evals\_ch6)}
\FunctionTok{summary}\NormalTok{(score\_model\_parallel\_slopes)}
\end{Highlighting}
\end{Shaded}

\begin{verbatim}

Call:
lm(formula = score ~ age + gender, data = evals_ch6)

Residuals:
     Min       1Q   Median       3Q      Max 
-1.82833 -0.33494  0.09391  0.42882  0.91506 

Coefficients:
             Estimate Std. Error t value Pr(>|t|)    
(Intercept)  4.484116   0.125284  35.792  < 2e-16 ***
age         -0.008678   0.002646  -3.280 0.001117 ** 
gendermale   0.190571   0.052469   3.632 0.000313 ***
---
Signif. codes:  0 '***' 0.001 '**' 0.01 '*' 0.05 '.' 0.1 ' ' 1

Residual standard error: 0.5343 on 460 degrees of freedom
Multiple R-squared:  0.03901,   Adjusted R-squared:  0.03484 
F-statistic: 9.338 on 2 and 460 DF,  p-value: 0.0001059
\end{verbatim}

\normalsize
\end{frame}

\begin{frame}[fragile]{Compare Interaction and Parallel slopes model}
\protect\hypertarget{compare-interaction-and-parallel-slopes-model-2}{}
\footnotesize

\begin{Shaded}
\begin{Highlighting}[]
\CommentTok{\# Get summaries of models:}
\FunctionTok{get\_regression\_summaries}\NormalTok{(score\_model\_interaction)}
\end{Highlighting}
\end{Shaded}

\begin{verbatim}
# A tibble: 1 x 9
  r_squared adj_r_squared   mse  rmse sigma statistic p_value    df  nobs
      <dbl>         <dbl> <dbl> <dbl> <dbl>     <dbl>   <dbl> <dbl> <dbl>
1     0.051         0.045 0.280 0.529 0.531      8.29       0     3   463
\end{verbatim}

\begin{Shaded}
\begin{Highlighting}[]
\FunctionTok{get\_regression\_summaries}\NormalTok{(score\_model\_parallel\_slopes)}
\end{Highlighting}
\end{Shaded}

\begin{verbatim}
# A tibble: 1 x 9
  r_squared adj_r_squared   mse  rmse sigma statistic p_value    df  nobs
      <dbl>         <dbl> <dbl> <dbl> <dbl>     <dbl>   <dbl> <dbl> <dbl>
1     0.039         0.035 0.284 0.533 0.534      9.34       0     2   463
\end{verbatim}

\normalsize
\end{frame}

\begin{frame}{AIC: Akaike Information Criterion}
\protect\hypertarget{aic-akaike-information-criterion}{}
Akaike's information criterion (AIC) can be motivated in two ways. The
most popular motivation seems to be based on balancing goodness of fit
and a penalty for model complexity. \textbf{AIC is defined such that the
smaller the value of AIC the better the model.}

\[AIC=n\log(SSE_{m}/n)+2m.\]

Recall that \(m\) is the number of parameters in your subset model. For
example, if your model includes only \(\beta_0, \beta_1, \beta_2\), then
\(m = 3\).

\textbf{Caution}: When the sample size is small, or when the number of
parameters estimated is a moderate to large fraction of the sample size,
it is well-known that AIC has a tendency for over-fitting since the
penalty for model complexity is not strong enough.
\end{frame}

\begin{frame}{BIC: Bayes Information Criterion}
\protect\hypertarget{bic-bayes-information-criterion}{}
BIC is defined such that the smaller the value of BIC the better the
model. \[BIC=n\log (SSE_m/n)+m\log(n)\] BIC penalizes complex models
more heavily than AIC, thus favoring simpler models than AIC.
\end{frame}

\begin{frame}[fragile]{Compare Interaction and Parallel slopes model}
\protect\hypertarget{compare-interaction-and-parallel-slopes-model-3}{}
\normalsize

\begin{Shaded}
\begin{Highlighting}[]
\CommentTok{\# AIC}
\FunctionTok{AIC}\NormalTok{(score\_model\_interaction)}
\end{Highlighting}
\end{Shaded}

\begin{verbatim}
[1] 734.5273
\end{verbatim}

\begin{Shaded}
\begin{Highlighting}[]
\FunctionTok{AIC}\NormalTok{(score\_model\_parallel\_slopes)}
\end{Highlighting}
\end{Shaded}

\begin{verbatim}
[1] 738.5253
\end{verbatim}

\begin{Shaded}
\begin{Highlighting}[]
\CommentTok{\# BIC}
\FunctionTok{BIC}\NormalTok{(score\_model\_interaction)}
\end{Highlighting}
\end{Shaded}

\begin{verbatim}
[1] 755.2159
\end{verbatim}

\begin{Shaded}
\begin{Highlighting}[]
\FunctionTok{BIC}\NormalTok{(score\_model\_parallel\_slopes)}
\end{Highlighting}
\end{Shaded}

\begin{verbatim}
[1] 755.0762
\end{verbatim}

\normalsize
\end{frame}

\begin{frame}{``Best'' Subset Algorithm}
\protect\hypertarget{best-subset-algorithm}{}
Time-saving algorithms have been developed in which the best subsets
according to a specified criterion are identified without requiring the
fitting of all of possible subset regression models.

Example: For the eight predictors, we know there are \(2^8=256\)
possible models.
\end{frame}

\begin{frame}[fragile]{Stepwise Procedures}
\protect\hypertarget{stepwise-procedures}{}
\textbf{Forward selection}

Forward selection starts with no variables in the model and then adds
the \(x-\)variable that produces the smallest \(\wp\text{-value}\) below
\(\alpha_{\text{crit}}\) when included in the model. This procedure is
continued until no new predictors can be added. The user can determine
the variable that produces the smallest \(\wp\text{-value}\) by
regressing the response variable on the \(x_i\)s one at a time using
\texttt{lm()} and \texttt{summary()} or using the \texttt{add()}
function.
\end{frame}

\begin{frame}[fragile]{Stepwise Procedures}
\protect\hypertarget{stepwise-procedures-1}{}
\textbf{Backward elimination}

Backward elimination begins with a model contining all potential \(x-\)
variables and identifies the one with the largest \(\wp\text{-value}\).
This can be done by looking at the \(\wp\text{-value}\)s for the \(t-\)
values of the \(\hat{\beta_i}, i = 1, \ldots,p-1\) using the function
\texttt{summary()} or by using the \(\wp\text{-value}\)s from the
function \texttt{drop1()}. If the variable with the largest
\(\wp\text{-value}\) is a above a predetermined value,
\(\alpha_{\text{crit}}\), that variable is dropped. A model with the
remaining \(x-\)variables is then fit and the procedure continues until
all the \(\wp\text{-value}\)s for the remaining variables in the model
are below the predetermined \(\alpha_{\text{crit}}\). The
\(\alpha_{\text{crit}}\) is sometimes referred to as the
``\(\wp\text{-value}\)-to-remove'' and is typically set to 15 or 20\%.
\end{frame}

\begin{frame}{Term Life Insurance Example}
\protect\hypertarget{term-life-insurance-example-4}{}
Like all firms, life insurance companies continually seek new ways to
deliver products to the market. Those involved in product development
want to know who buys insurance and how much they buy. In this example,
we examine the Survey of Consumer Finances (SCF), that contains
extensive information on assets, liabilities, income, and demographic
characteristics of those sampled (potential U.S. customers). We study a
random sample of 500 households with positive incomes that were
interviewed in the 2004 survey.
\end{frame}

\begin{frame}[fragile]{Term Life Insurance Example}
\protect\hypertarget{term-life-insurance-example-5}{}
\tiny

\begin{Shaded}
\begin{Highlighting}[]
\DocumentationTok{\#\#\#\#forward selection based on AIC \#\#\#\#\#}
\FunctionTok{library}\NormalTok{(MASS)}
\NormalTok{null }\OtherTok{\textless{}{-}} \FunctionTok{lm}\NormalTok{(ln\_face  }\SpecialCharTok{\textasciitilde{}} \DecValTok{1}\NormalTok{, }\AttributeTok{data =}\NormalTok{ term)}
\NormalTok{full }\OtherTok{\textless{}{-}} \FunctionTok{lm}\NormalTok{(ln\_face }\SpecialCharTok{\textasciitilde{}}\NormalTok{ ., }\AttributeTok{data =}\NormalTok{ term)}
\NormalTok{mod\_fs }\OtherTok{\textless{}{-}} \FunctionTok{stepAIC}\NormalTok{(null, }\AttributeTok{scope =} \FunctionTok{list}\NormalTok{(}\AttributeTok{lower=}\NormalTok{ null, }\AttributeTok{upper=}\NormalTok{ full), }
                  \AttributeTok{direction =} \StringTok{"forward"}\NormalTok{, }\AttributeTok{trace =} \DecValTok{0}\NormalTok{)}
\FunctionTok{summary}\NormalTok{(mod\_fs)}
\end{Highlighting}
\end{Shaded}

\begin{verbatim}

Call:
lm(formula = ln_face ~ ln_income + education + numhh, data = term)

Residuals:
    Min      1Q  Median      3Q     Max 
-5.7420 -0.8681  0.0549  0.9093  4.7187 

Coefficients:
            Estimate Std. Error t value Pr(>|t|)    
(Intercept)  2.58408    0.84643   3.053  0.00249 ** 
ln_income    0.49353    0.07754   6.365 8.32e-10 ***
education    0.20641    0.03883   5.316 2.22e-07 ***
numhh        0.30605    0.06333   4.833 2.26e-06 ***
---
Signif. codes:  0 '***' 0.001 '**' 0.01 '*' 0.05 '.' 0.1 ' ' 1

Residual standard error: 1.525 on 271 degrees of freedom
Multiple R-squared:  0.3425,    Adjusted R-squared:  0.3353 
F-statistic: 47.07 on 3 and 271 DF,  p-value: < 2.2e-16
\end{verbatim}

\normalsize
\end{frame}

\begin{frame}[fragile]{Term Life Insurance Example}
\protect\hypertarget{term-life-insurance-example-6}{}
\scriptsize

\begin{Shaded}
\begin{Highlighting}[]
\DocumentationTok{\#\#\#\# backward elimination based on AIC \#\#\#\#\#}
\NormalTok{mod\_be }\OtherTok{\textless{}{-}} \FunctionTok{stepAIC}\NormalTok{(full, }\AttributeTok{scope =} \FunctionTok{list}\NormalTok{(}\AttributeTok{lower =}\NormalTok{ null, }\AttributeTok{upper =}\NormalTok{ full), }
                 \AttributeTok{direction =} \StringTok{"backward"}\NormalTok{, }\AttributeTok{trace =} \DecValTok{0}\NormalTok{)}
\FunctionTok{summary}\NormalTok{(mod\_be)}
\end{Highlighting}
\end{Shaded}

\begin{verbatim}

Call:
lm(formula = ln_face ~ education + ln_income + numhh, data = term)

Residuals:
    Min      1Q  Median      3Q     Max 
-5.7420 -0.8681  0.0549  0.9093  4.7187 

Coefficients:
            Estimate Std. Error t value Pr(>|t|)    
(Intercept)  2.58408    0.84643   3.053  0.00249 ** 
education    0.20641    0.03883   5.316 2.22e-07 ***
ln_income    0.49353    0.07754   6.365 8.32e-10 ***
numhh        0.30605    0.06333   4.833 2.26e-06 ***
---
Signif. codes:  0 '***' 0.001 '**' 0.01 '*' 0.05 '.' 0.1 ' ' 1

Residual standard error: 1.525 on 271 degrees of freedom
Multiple R-squared:  0.3425,    Adjusted R-squared:  0.3353 
F-statistic: 47.07 on 3 and 271 DF,  p-value: < 2.2e-16
\end{verbatim}

\normalsize
\end{frame}

\hypertarget{multicollinearity}{%
\section{Multicollinearity}\label{multicollinearity}}

\begin{frame}{Multicollinearity and its Effects}
\protect\hypertarget{multicollinearity-and-its-effects}{}
\begin{itemize}
\item
  \textbf{Multicollinearity} exists when two or more of the predictors
  in a regression model are moderately or highly correlated with one
  another.

  \begin{itemize}
  \tightlist
  \item
    Unfortunately, when it exists, it can wreak havoc on our analysis
    and thereby limit the research conclusions we can draw.
  \end{itemize}
\item
  When multicollinearity exists, any of the following outcomes can be
  exacerbated:

  \begin{itemize}
  \tightlist
  \item
    The estimated regression coefficient of any one variable depends on
    which other predictors are included in the model.
  \item
    The standard errors and hence the variances of the estimated
    coefficients are inflated when multicollinearity exists.
  \item
    Inflated variances impact the conclusion for hypothesis tests for
    \(\beta_k = 0\).
  \end{itemize}
\end{itemize}
\end{frame}

\begin{frame}{Variance Inflation Factor}
\protect\hypertarget{variance-inflation-factor}{}
\begin{itemize}
\item
  The \textbf{variance inflation factors (VIF)} quantifies how much the
  variance of the estimated coefficients are inflated.
\item
  Hence, the variance inflation factor for the estimated regression
  coefficient \(b_j\), denoted \(VIF_j\) is just the factor by which the
  variance of \(b_j\) is ``inflated'' by the existence of correlation
  among the predictor variables in the model.
\end{itemize}
\end{frame}

\begin{frame}{Variance Inflation Factor}
\protect\hypertarget{variance-inflation-factor-1}{}
In particular, the variance inflation factor for the \(j\)th predictor
is

\[VIF_j=\frac{1}{1-R^2_j}\]

where \(R^2_j\) is the \(R^2\)-value obtained by regressing the \(j\)th
predictor on the remaining predictors.

\begin{itemize}
\item
  How do we interpret the variance inflation factors for a regression
  model?

  \begin{itemize}
  \tightlist
  \item
    A VIF of 1 means that there is no correlation among the \(j\)th
    predictor and the remaining predictor variables, and hence the
    variance of \(b_j\) is not inflated at all.
  \item
    The general rule of thumb is that VIFs exceeding 4 warrant further
    investigation, while VIFs exceeding 10 are signs of serious
    multicollinearity requiring correction.
  \end{itemize}
\end{itemize}
\end{frame}

\begin{frame}{High Blood Pressure Example}
\protect\hypertarget{high-blood-pressure-example}{}
\begin{itemize}
\item
  The researchers were interested in determining if a relationship
  exists between blood pressure and age, weight, body surface area,
  duration, pulse rate and stress level.

  \begin{itemize}
  \tightlist
  \item
    blood pressure (y = bp, in mm Hg)
  \item
    age (x1 = age, in years)
  \item
    weight (x2 = weight, in kg)
  \item
    body surface area (x3 = bsa, in sq m)
  \item
    duration of hypertension (x4 = dur, in years)
  \item
    basal pulse (x5 = pulse, in beats per minute)
  \item
    stress index (x6 = stress)
  \end{itemize}
\end{itemize}
\end{frame}

\begin{frame}[fragile]{High Blood Pressure Example}
\protect\hypertarget{high-blood-pressure-example-1}{}
High correlation between \texttt{weight} and \texttt{bsa}

\scriptsize

\begin{Shaded}
\begin{Highlighting}[]
\NormalTok{bloodpress }\OtherTok{\textless{}{-}} \FunctionTok{read.csv}\NormalTok{(}\StringTok{"bloodpress.csv"}\NormalTok{)}
\NormalTok{bloodpress }\OtherTok{\textless{}{-}}\NormalTok{ bloodpress }\SpecialCharTok{\%\textgreater{}\%} 
  \FunctionTok{clean\_names}\NormalTok{()}
\NormalTok{bloodpress }\OtherTok{\textless{}{-}}\NormalTok{ bloodpress[, }\SpecialCharTok{{-}}\DecValTok{1}\NormalTok{]}
\FunctionTok{cor}\NormalTok{(bloodpress, }\AttributeTok{use =} \StringTok{"complete.obs"}\NormalTok{)}
\end{Highlighting}
\end{Shaded}

\begin{verbatim}
              bp       age     weight        bsa       dur     pulse     stress
bp     1.0000000 0.6590930 0.95006765 0.86587887 0.2928336 0.7214132 0.16390139
age    0.6590930 1.0000000 0.40734926 0.37845460 0.3437921 0.6187643 0.36822369
weight 0.9500677 0.4073493 1.00000000 0.87530481 0.2006496 0.6593399 0.03435475
bsa    0.8658789 0.3784546 0.87530481 1.00000000 0.1305400 0.4648188 0.01844634
dur    0.2928336 0.3437921 0.20064959 0.13054001 1.0000000 0.4015144 0.31163982
pulse  0.7214132 0.6187643 0.65933987 0.46481881 0.4015144 1.0000000 0.50631008
stress 0.1639014 0.3682237 0.03435475 0.01844634 0.3116398 0.5063101 1.00000000
\end{verbatim}

\normalsize
\end{frame}

\begin{frame}[fragile]{High Blood Pressure Example}
\protect\hypertarget{high-blood-pressure-example-2}{}
\scriptsize

\begin{Shaded}
\begin{Highlighting}[]
\NormalTok{model\_bp }\OtherTok{\textless{}{-}} \FunctionTok{lm}\NormalTok{(bp }\SpecialCharTok{\textasciitilde{}}\NormalTok{ ., }\AttributeTok{data =}\NormalTok{ bloodpress)}
\FunctionTok{summary}\NormalTok{(model\_bp)}
\end{Highlighting}
\end{Shaded}

\begin{verbatim}

Call:
lm(formula = bp ~ ., data = bloodpress)

Residuals:
     Min       1Q   Median       3Q      Max 
-0.93213 -0.11314  0.03064  0.21834  0.48454 

Coefficients:
              Estimate Std. Error t value Pr(>|t|)    
(Intercept) -12.870476   2.556650  -5.034 0.000229 ***
age           0.703259   0.049606  14.177 2.76e-09 ***
weight        0.969920   0.063108  15.369 1.02e-09 ***
bsa           3.776491   1.580151   2.390 0.032694 *  
dur           0.068383   0.048441   1.412 0.181534    
pulse        -0.084485   0.051609  -1.637 0.125594    
stress        0.005572   0.003412   1.633 0.126491    
---
Signif. codes:  0 '***' 0.001 '**' 0.01 '*' 0.05 '.' 0.1 ' ' 1

Residual standard error: 0.4072 on 13 degrees of freedom
Multiple R-squared:  0.9962,    Adjusted R-squared:  0.9944 
F-statistic: 560.6 on 6 and 13 DF,  p-value: 6.395e-15
\end{verbatim}

\normalsize
\end{frame}

\begin{frame}[fragile]{High Blood Pressure Example}
\protect\hypertarget{high-blood-pressure-example-3}{}
\normalsize

\begin{Shaded}
\begin{Highlighting}[]
\FunctionTok{library}\NormalTok{(car)}
\FunctionTok{vif}\NormalTok{(model\_bp)}
\end{Highlighting}
\end{Shaded}

\begin{verbatim}
     age   weight      bsa      dur    pulse   stress 
1.762807 8.417035 5.328751 1.237309 4.413575 1.834845 
\end{verbatim}

\normalsize

\begin{itemize}
\item
  The \(\text{VIF}_j\) for a predictor \(x_j\) can be interpreted as the
  factor (\(\sqrt{\text{VIF}_j}\)) by which the standard error of
  \(\hat{\beta}_j\) is increased due to the presence of
  multicollinearity.
\item
  Regressing \texttt{weight} on the remaining five predictors, gives
  \(R^2_{\text{weight}}=88.12\%\).
  \[VIF_{\text{weight}}=\frac{1}{1-R^2_{\text{weight}}}=\frac{1}{1-0.8812}=8.42\]
\end{itemize}
\end{frame}

\begin{frame}[fragile]{Variance Inflation Factor}
\protect\hypertarget{variance-inflation-factor-2}{}
\scriptsize

\begin{Shaded}
\begin{Highlighting}[]
\NormalTok{r2age }\OtherTok{\textless{}{-}} \FunctionTok{summary}\NormalTok{(}\FunctionTok{lm}\NormalTok{(age }\SpecialCharTok{\textasciitilde{}}\NormalTok{ weight }\SpecialCharTok{+}\NormalTok{ bsa }\SpecialCharTok{+}\NormalTok{ dur }\SpecialCharTok{+}\NormalTok{ pulse }\SpecialCharTok{+} 
\NormalTok{                      stress, }\AttributeTok{data =}\NormalTok{ bloodpress))}\SpecialCharTok{$}\NormalTok{r.squared}
\NormalTok{r2weight }\OtherTok{\textless{}{-}} \FunctionTok{summary}\NormalTok{(}\FunctionTok{lm}\NormalTok{(weight }\SpecialCharTok{\textasciitilde{}}\NormalTok{ age }\SpecialCharTok{+}\NormalTok{ bsa }\SpecialCharTok{+}\NormalTok{ dur }\SpecialCharTok{+}\NormalTok{ pulse }\SpecialCharTok{+} 
\NormalTok{                         stress, }\AttributeTok{data =}\NormalTok{ bloodpress))}\SpecialCharTok{$}\NormalTok{r.squared}
\NormalTok{r2bsa }\OtherTok{\textless{}{-}} \FunctionTok{summary}\NormalTok{(}\FunctionTok{lm}\NormalTok{(bsa }\SpecialCharTok{\textasciitilde{}}\NormalTok{ age }\SpecialCharTok{+}\NormalTok{ weight }\SpecialCharTok{+}\NormalTok{ dur }\SpecialCharTok{+}\NormalTok{ pulse }\SpecialCharTok{+} 
\NormalTok{                      stress, }\AttributeTok{data =}\NormalTok{ bloodpress))}\SpecialCharTok{$}\NormalTok{r.squared}
\NormalTok{r2dur }\OtherTok{\textless{}{-}} \FunctionTok{summary}\NormalTok{(}\FunctionTok{lm}\NormalTok{(dur }\SpecialCharTok{\textasciitilde{}}\NormalTok{ age }\SpecialCharTok{+}\NormalTok{ weight }\SpecialCharTok{+}\NormalTok{ bsa }\SpecialCharTok{+}\NormalTok{ pulse }\SpecialCharTok{+} 
\NormalTok{                      stress, }\AttributeTok{data =}\NormalTok{ bloodpress))}\SpecialCharTok{$}\NormalTok{r.squared}
\NormalTok{r2pulse }\OtherTok{\textless{}{-}} \FunctionTok{summary}\NormalTok{(}\FunctionTok{lm}\NormalTok{(pulse }\SpecialCharTok{\textasciitilde{}}\NormalTok{ age }\SpecialCharTok{+}\NormalTok{ weight }\SpecialCharTok{+}\NormalTok{ bsa }\SpecialCharTok{+}\NormalTok{ dur }\SpecialCharTok{+} 
\NormalTok{                        stress, }\AttributeTok{data =}\NormalTok{ bloodpress))}\SpecialCharTok{$}\NormalTok{r.squared}
\NormalTok{r2stress }\OtherTok{\textless{}{-}} \FunctionTok{summary}\NormalTok{(}\FunctionTok{lm}\NormalTok{(stress }\SpecialCharTok{\textasciitilde{}}\NormalTok{ age }\SpecialCharTok{+}\NormalTok{ weight }\SpecialCharTok{+}\NormalTok{ bsa }\SpecialCharTok{+}\NormalTok{ dur }\SpecialCharTok{+} 
\NormalTok{                         pulse, }\AttributeTok{data =}\NormalTok{ bloodpress))}\SpecialCharTok{$}\NormalTok{r.squared}
\FunctionTok{c}\NormalTok{(r2age, r2weight, r2bsa, r2dur, r2pulse, r2stress, r2age) }\OtherTok{{-}\textgreater{}}\NormalTok{ r2s}
\NormalTok{r2s}
\end{Highlighting}
\end{Shaded}

\begin{verbatim}
[1] 0.4327228 0.8811933 0.8123388 0.1917947 0.7734263 0.4549949 0.4327228
\end{verbatim}

\begin{Shaded}
\begin{Highlighting}[]
\NormalTok{(VIFs }\OtherTok{\textless{}{-}} \DecValTok{1} \SpecialCharTok{/}\NormalTok{(}\DecValTok{1} \SpecialCharTok{{-}}\NormalTok{ r2s))}
\end{Highlighting}
\end{Shaded}

\begin{verbatim}
[1] 1.762807 8.417035 5.328751 1.237309 4.413575 1.834845 1.762807
\end{verbatim}

\begin{Shaded}
\begin{Highlighting}[]
\FunctionTok{sqrt}\NormalTok{(VIFs)}
\end{Highlighting}
\end{Shaded}

\begin{verbatim}
[1] 1.327707 2.901213 2.308409 1.112344 2.100851 1.354565 1.327707
\end{verbatim}

\normalsize
\end{frame}

\begin{frame}[fragile]{Variance Inflation Factor}
\protect\hypertarget{variance-inflation-factor-3}{}
\tiny

\begin{Shaded}
\begin{Highlighting}[]
\FunctionTok{cor}\NormalTok{(bloodpress, }\AttributeTok{use =} \StringTok{"complete.obs"}\NormalTok{)}
\end{Highlighting}
\end{Shaded}

\begin{verbatim}
              bp       age     weight        bsa       dur     pulse     stress
bp     1.0000000 0.6590930 0.95006765 0.86587887 0.2928336 0.7214132 0.16390139
age    0.6590930 1.0000000 0.40734926 0.37845460 0.3437921 0.6187643 0.36822369
weight 0.9500677 0.4073493 1.00000000 0.87530481 0.2006496 0.6593399 0.03435475
bsa    0.8658789 0.3784546 0.87530481 1.00000000 0.1305400 0.4648188 0.01844634
dur    0.2928336 0.3437921 0.20064959 0.13054001 1.0000000 0.4015144 0.31163982
pulse  0.7214132 0.6187643 0.65933987 0.46481881 0.4015144 1.0000000 0.50631008
stress 0.1639014 0.3682237 0.03435475 0.01844634 0.3116398 0.5063101 1.00000000
\end{verbatim}

\normalsize

\begin{itemize}
\item
  We see that:

  \begin{itemize}
  \tightlist
  \item
    \texttt{weight} and \texttt{bsa} are highly correlated
    (\(r= 0.875\)).
  \item
    \texttt{pulse} also appears to exhibit fairly strong marginal
    correlations with several of the predictors, including \texttt{age}
    (\(r = 0.619\)), \texttt{weight} (\(r = 0.659\)) and \texttt{stress}
    (\(r = 0.506\))
  \end{itemize}
\item
  We will remove \texttt{bsa} and \texttt{pulse} from the data.
\end{itemize}
\end{frame}

\begin{frame}[fragile]{Variance Inflation Factor}
\protect\hypertarget{variance-inflation-factor-4}{}
\tiny

\begin{Shaded}
\begin{Highlighting}[]
\NormalTok{model\_bp\_new }\OtherTok{\textless{}{-}} \FunctionTok{lm}\NormalTok{(bp }\SpecialCharTok{\textasciitilde{}}\NormalTok{ age }\SpecialCharTok{+}\NormalTok{ weight }\SpecialCharTok{+}\NormalTok{ dur }\SpecialCharTok{+}\NormalTok{ stress, }\AttributeTok{data =}\NormalTok{ bloodpress)}
\FunctionTok{summary}\NormalTok{(model\_bp\_new)}
\end{Highlighting}
\end{Shaded}

\begin{verbatim}

Call:
lm(formula = bp ~ age + weight + dur + stress, data = bloodpress)

Residuals:
     Min       1Q   Median       3Q      Max 
-1.11359 -0.29586  0.01515  0.27506  0.88674 

Coefficients:
              Estimate Std. Error t value Pr(>|t|)    
(Intercept) -15.869829   3.195296  -4.967 0.000169 ***
age           0.683741   0.061195  11.173 1.14e-08 ***
weight        1.034128   0.032672  31.652 3.76e-15 ***
dur           0.039889   0.064486   0.619 0.545485    
stress        0.002184   0.003794   0.576 0.573304    
---
Signif. codes:  0 '***' 0.001 '**' 0.01 '*' 0.05 '.' 0.1 ' ' 1

Residual standard error: 0.5505 on 15 degrees of freedom
Multiple R-squared:  0.9919,    Adjusted R-squared:  0.9897 
F-statistic: 458.3 on 4 and 15 DF,  p-value: 1.764e-15
\end{verbatim}

\begin{Shaded}
\begin{Highlighting}[]
\FunctionTok{vif}\NormalTok{(model\_bp\_new)}
\end{Highlighting}
\end{Shaded}

\begin{verbatim}
     age   weight      dur   stress 
1.468245 1.234653 1.200060 1.241117 
\end{verbatim}

\normalsize
\end{frame}

\end{document}
