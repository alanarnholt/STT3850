% Options for packages loaded elsewhere
\PassOptionsToPackage{unicode}{hyperref}
\PassOptionsToPackage{hyphens}{url}
%
\documentclass[
  ignorenonframetext,
]{beamer}
\usepackage{pgfpages}
\setbeamertemplate{caption}[numbered]
\setbeamertemplate{caption label separator}{: }
\setbeamercolor{caption name}{fg=normal text.fg}
\beamertemplatenavigationsymbolsempty
% Prevent slide breaks in the middle of a paragraph
\widowpenalties 1 10000
\raggedbottom
\setbeamertemplate{part page}{
  \centering
  \begin{beamercolorbox}[sep=16pt,center]{part title}
    \usebeamerfont{part title}\insertpart\par
  \end{beamercolorbox}
}
\setbeamertemplate{section page}{
  \centering
  \begin{beamercolorbox}[sep=12pt,center]{part title}
    \usebeamerfont{section title}\insertsection\par
  \end{beamercolorbox}
}
\setbeamertemplate{subsection page}{
  \centering
  \begin{beamercolorbox}[sep=8pt,center]{part title}
    \usebeamerfont{subsection title}\insertsubsection\par
  \end{beamercolorbox}
}
\AtBeginPart{
  \frame{\partpage}
}
\AtBeginSection{
  \ifbibliography
  \else
    \frame{\sectionpage}
  \fi
}
\AtBeginSubsection{
  \frame{\subsectionpage}
}
\usepackage{amsmath,amssymb}
\usepackage{iftex}
\ifPDFTeX
  \usepackage[T1]{fontenc}
  \usepackage[utf8]{inputenc}
  \usepackage{textcomp} % provide euro and other symbols
\else % if luatex or xetex
  \usepackage{unicode-math} % this also loads fontspec
  \defaultfontfeatures{Scale=MatchLowercase}
  \defaultfontfeatures[\rmfamily]{Ligatures=TeX,Scale=1}
\fi
\usepackage{lmodern}
\usetheme[]{Madrid}
\usecolortheme{orchid}
\usefonttheme{professionalfonts}
\ifPDFTeX\else
  % xetex/luatex font selection
\fi
% Use upquote if available, for straight quotes in verbatim environments
\IfFileExists{upquote.sty}{\usepackage{upquote}}{}
\IfFileExists{microtype.sty}{% use microtype if available
  \usepackage[]{microtype}
  \UseMicrotypeSet[protrusion]{basicmath} % disable protrusion for tt fonts
}{}
\makeatletter
\@ifundefined{KOMAClassName}{% if non-KOMA class
  \IfFileExists{parskip.sty}{%
    \usepackage{parskip}
  }{% else
    \setlength{\parindent}{0pt}
    \setlength{\parskip}{6pt plus 2pt minus 1pt}}
}{% if KOMA class
  \KOMAoptions{parskip=half}}
\makeatother
\usepackage{xcolor}
\newif\ifbibliography
\usepackage{color}
\usepackage{fancyvrb}
\newcommand{\VerbBar}{|}
\newcommand{\VERB}{\Verb[commandchars=\\\{\}]}
\DefineVerbatimEnvironment{Highlighting}{Verbatim}{commandchars=\\\{\}}
% Add ',fontsize=\small' for more characters per line
\usepackage{framed}
\definecolor{shadecolor}{RGB}{248,248,248}
\newenvironment{Shaded}{\begin{snugshade}}{\end{snugshade}}
\newcommand{\AlertTok}[1]{\textcolor[rgb]{0.94,0.16,0.16}{#1}}
\newcommand{\AnnotationTok}[1]{\textcolor[rgb]{0.56,0.35,0.01}{\textbf{\textit{#1}}}}
\newcommand{\AttributeTok}[1]{\textcolor[rgb]{0.13,0.29,0.53}{#1}}
\newcommand{\BaseNTok}[1]{\textcolor[rgb]{0.00,0.00,0.81}{#1}}
\newcommand{\BuiltInTok}[1]{#1}
\newcommand{\CharTok}[1]{\textcolor[rgb]{0.31,0.60,0.02}{#1}}
\newcommand{\CommentTok}[1]{\textcolor[rgb]{0.56,0.35,0.01}{\textit{#1}}}
\newcommand{\CommentVarTok}[1]{\textcolor[rgb]{0.56,0.35,0.01}{\textbf{\textit{#1}}}}
\newcommand{\ConstantTok}[1]{\textcolor[rgb]{0.56,0.35,0.01}{#1}}
\newcommand{\ControlFlowTok}[1]{\textcolor[rgb]{0.13,0.29,0.53}{\textbf{#1}}}
\newcommand{\DataTypeTok}[1]{\textcolor[rgb]{0.13,0.29,0.53}{#1}}
\newcommand{\DecValTok}[1]{\textcolor[rgb]{0.00,0.00,0.81}{#1}}
\newcommand{\DocumentationTok}[1]{\textcolor[rgb]{0.56,0.35,0.01}{\textbf{\textit{#1}}}}
\newcommand{\ErrorTok}[1]{\textcolor[rgb]{0.64,0.00,0.00}{\textbf{#1}}}
\newcommand{\ExtensionTok}[1]{#1}
\newcommand{\FloatTok}[1]{\textcolor[rgb]{0.00,0.00,0.81}{#1}}
\newcommand{\FunctionTok}[1]{\textcolor[rgb]{0.13,0.29,0.53}{\textbf{#1}}}
\newcommand{\ImportTok}[1]{#1}
\newcommand{\InformationTok}[1]{\textcolor[rgb]{0.56,0.35,0.01}{\textbf{\textit{#1}}}}
\newcommand{\KeywordTok}[1]{\textcolor[rgb]{0.13,0.29,0.53}{\textbf{#1}}}
\newcommand{\NormalTok}[1]{#1}
\newcommand{\OperatorTok}[1]{\textcolor[rgb]{0.81,0.36,0.00}{\textbf{#1}}}
\newcommand{\OtherTok}[1]{\textcolor[rgb]{0.56,0.35,0.01}{#1}}
\newcommand{\PreprocessorTok}[1]{\textcolor[rgb]{0.56,0.35,0.01}{\textit{#1}}}
\newcommand{\RegionMarkerTok}[1]{#1}
\newcommand{\SpecialCharTok}[1]{\textcolor[rgb]{0.81,0.36,0.00}{\textbf{#1}}}
\newcommand{\SpecialStringTok}[1]{\textcolor[rgb]{0.31,0.60,0.02}{#1}}
\newcommand{\StringTok}[1]{\textcolor[rgb]{0.31,0.60,0.02}{#1}}
\newcommand{\VariableTok}[1]{\textcolor[rgb]{0.00,0.00,0.00}{#1}}
\newcommand{\VerbatimStringTok}[1]{\textcolor[rgb]{0.31,0.60,0.02}{#1}}
\newcommand{\WarningTok}[1]{\textcolor[rgb]{0.56,0.35,0.01}{\textbf{\textit{#1}}}}
\usepackage{longtable,booktabs,array}
\usepackage{calc} % for calculating minipage widths
\usepackage{caption}
% Make caption package work with longtable
\makeatletter
\def\fnum@table{\tablename~\thetable}
\makeatother
\setlength{\emergencystretch}{3em} % prevent overfull lines
\providecommand{\tightlist}{%
  \setlength{\itemsep}{0pt}\setlength{\parskip}{0pt}}
\setcounter{secnumdepth}{-\maxdimen} % remove section numbering
\usepackage{placeins}
\usepackage{color}
\usepackage{bm}
\usepackage{amsmath}
\usepackage{algorithm}
\usepackage[]{algpseudocode}
\usepackage{tabularx}
\usepackage{multirow}
\usepackage[most]{tcolorbox}
\usepackage{tikz}
\usepackage{lipsum}
\usepackage{mathtools}
\usepackage{actuarialangle}
\usepackage{multirow, longtable, array, dcolumn}
\usepackage{tabu}
\newcommand{\sdt}{\bullet}
\newcommand{\tss}{\textsuperscript}
\newcommand{\morearraysp}{\setlength{\arraycolsep}{2mm}}
\newcommand{\smarraysp}{\setlength{\arraycolsep}{1mm}}
\newcommand{\oldarraysp}{\setlength{\arraycolsep}{1.5pt}}
\newcommand{\matrixstretch}{\setlength{\extrarowheight}{4pt}}
\newcommand{\matrixnostretch}{\setlength{\extrarowheight}{0pt}}
\newcommand{\gil}[1]{\textrm{\gilfont{#1}}\normalfont }
\newfont{\gilfont}{msbm10 scaled 1000}
\newcommand{\DOT}{\usebox{\biggercirc}}
\newcommand{\pv}{\wp\text{-value}}
\ifLuaTeX
  \usepackage{selnolig}  % disable illegal ligatures
\fi
\IfFileExists{bookmark.sty}{\usepackage{bookmark}}{\usepackage{hyperref}}
\IfFileExists{xurl.sty}{\usepackage{xurl}}{} % add URL line breaks if available
\urlstyle{same}
\hypersetup{
  pdftitle={STT 3850 : Week 4},
  pdfauthor={Spring 2024},
  hidelinks,
  pdfcreator={LaTeX via pandoc}}

\title{STT 3850 : Week 4}
\author{Spring 2024}
\date{}
\institute{Appalachian State University}

\begin{document}
\frame{\titlepage}

\hypertarget{outline-for-the-week}{%
\section{Outline for the week}\label{outline-for-the-week}}

\begin{frame}{By the end of the week: Basic Regression}
\protect\hypertarget{by-the-end-of-the-week-basic-regression}{}
\begin{itemize}
\tightlist
\item
  Data Modeling
\item
  Exploratory data analysis
\item
  Linear regression
\end{itemize}
\end{frame}

\hypertarget{basic-regression}{%
\section{Basic Regression}\label{basic-regression}}

\begin{frame}{Basic Regression}
\protect\hypertarget{basic-regression-1}{}
\begin{itemize}
\item
  Now that we are equipped with

  \begin{itemize}
  \tightlist
  \item
    an understanding of how to import data
  \item
    data visualization and
  \item
    data wrangling skill
  \end{itemize}
\item
  Let's now proceed with \textbf{data modeling}.
\item
  The fundamental premise of data modeling is to make explicit the
  relationship between:

  \begin{itemize}
  \tightlist
  \item
    an \textbf{outcome variable} \(y\), also called a \textbf{dependent
    variable} or \textbf{response variable}, and
  \item
    an e\textbf{xplanatory/predictor} variable \(x\), also called an
    \textbf{independent variable} or \textbf{covariate}.
  \end{itemize}
\end{itemize}
\end{frame}

\begin{frame}{Data Modeling}
\protect\hypertarget{data-modeling}{}
Data modeling serves one of two purposes:

\begin{enumerate}
\item
  Modeling for explanation:

  \begin{itemize}
  \tightlist
  \item
    Describe and quantify the relationship between the outcome variable
    \(y\) and a set of explanatory variables \(x\).
  \item
    Determine the significance of any relationships.
  \item
    Have measures summarizing these relationships.
  \item
    Possibly identify any causal relationships between the variables.
  \end{itemize}
\item
  Modeling for prediction:

  \begin{itemize}
  \tightlist
  \item
    Predict an outcome variable \(y\) based on the information contained
    in a set of predictor variables \(x\).
  \item
    Here, you don't care so much about understanding how all the
    variables relate and interact with one another.
  \end{itemize}
\end{enumerate}
\end{frame}

\begin{frame}{Data Modeling}
\protect\hypertarget{data-modeling-1}{}
\begin{itemize}
\item
  For example, say you are interested in

  \begin{itemize}
  \tightlist
  \item
    an outcome variable \(y\) of whether patients develop lung cancer
    and
  \item
    information \(x\) on their risk factors, such as smoking habits,
    age, and socioeconomic status.
  \end{itemize}
\item
  If we are modeling for explanation,

  \begin{itemize}
  \tightlist
  \item
    we would be interested in both describing and quantifying the
    effects of the different risk factors.
  \item
    One reason could be that you want to design an intervention to
    reduce lung cancer incidence in a population, such as targeting
    smokers of a specific age group with advertising for smoking
    cessation programs.
  \end{itemize}
\item
  If we are modeling for prediction,

  \begin{itemize}
  \tightlist
  \item
    we wouldn't care so much about understanding how all the individual
    risk factors contribute to lung cancer,
  \item
    but rather only whether we can make good predictions of which people
    will contract lung cancer.
  \end{itemize}
\end{itemize}
\end{frame}

\begin{frame}{Linear regression}
\protect\hypertarget{linear-regression}{}
\begin{itemize}
\item
  There are many techniques for modeling, such as

  \begin{itemize}
  \tightlist
  \item
    tree-based models and
  \item
    neural networks,
  \end{itemize}
\item
  But in this class, we'll focus on one particular technique:
  \textbf{linear regression}.
\item
  Linear regression involves a numerical outcome variable \(y\) and
  explanatory variables \(x\) that are either numerical or categorical.

  \begin{itemize}
  \tightlist
  \item
    the relationship between \(y\) and \(x\) is assumed to be linear, or
    in other words, a line.
  \item
    However, we'll see that what constitutes a ``line'' will vary
    depending on the nature of your explanatory variables \(x\).
  \item
    Linear regression is one of the most commonly-used and
    easy-to-understand approaches to modeling.
  \end{itemize}
\end{itemize}
\end{frame}

\begin{frame}[fragile]{Needed packages}
\protect\hypertarget{needed-packages}{}
Let's now load all the packages needed

\normalsize

\begin{Shaded}
\begin{Highlighting}[]
\FunctionTok{library}\NormalTok{(ggplot2)    }\CommentTok{\#  for data visualization}
\FunctionTok{library}\NormalTok{(dplyr)      }\CommentTok{\#  for data wrangling}
\FunctionTok{library}\NormalTok{(readr)      }\CommentTok{\# for importing spreadsheet data into R}
\FunctionTok{library}\NormalTok{(moderndive) }\CommentTok{\# package of datasets and regression functions}
\FunctionTok{library}\NormalTok{(skimr)      }\CommentTok{\# provides simple{-}to{-}use functions }
                    \CommentTok{\# for summary statistics}
\end{Highlighting}
\end{Shaded}

\normalsize
\end{frame}

\begin{frame}[fragile]{One numerical explanatory variable}
\protect\hypertarget{one-numerical-explanatory-variable}{}
\begin{itemize}
\item
  Researchers at the University of Texas in Austin, Texas (UT Austin)
  tried to answer the following research question:

  \begin{itemize}
  \tightlist
  \item
    what factors explain differences in instructor teaching evaluation
    scores?
  \end{itemize}
\item
  To this end, they collected instructor and course information on 463
  courses.
\item
  A full description of the study can be found at
  \url{https://openintro.org}.
\item
  The data on the \texttt{463} courses at UT Austin can be found in the
  \texttt{evals} data frame included in the \texttt{moderndive} package.
\end{itemize}
\end{frame}

\begin{frame}[fragile]{One numerical explanatory variable}
\protect\hypertarget{one-numerical-explanatory-variable-1}{}
Let's fully describe the 4 variables we will focus on:

\begin{enumerate}
\item
  \texttt{ID}: An identification variable used to distinguish between
  the 1 through 463 courses in the dataset.
\item
  \texttt{score}: A numerical variable of the course instructor's
  average teaching score, where the average is computed from the
  evaluation scores from all students in that course. Teaching scores of
  1 are lowest and 5 are highest. This is the outcome variable \(y\) of
  interest.
\item
  \texttt{bty\_avg}: A numerical variable of the course instructor's
  average ``beauty'' score, where the average is computed from a
  separate panel of six students. ``Beauty'' scores of 1 are lowest and
  10 are highest. This is the explanatory variable \(x\) of interest.
\item
  \texttt{age}: A numerical variable of the course instructor's age.
  This will be another explanatory variable \(x\) that we'll use later.
\end{enumerate}
\end{frame}

\begin{frame}{One numerical explanatory variable}
\protect\hypertarget{one-numerical-explanatory-variable-2}{}
We'll answer these questions by modeling the relationship between
teaching scores and ``beauty'' scores using simple linear regression
where we have:

\begin{enumerate}
\item
  A numerical outcome variable \(y\) (the instructor's teaching score)
  and
\item
  A single numerical explanatory variable \(x\) (the instructor's
  ``beauty'' score).
\end{enumerate}
\end{frame}

\begin{frame}{Exploratory data analysis}
\protect\hypertarget{exploratory-data-analysis}{}
\begin{itemize}
\item
  A crucial step before doing any kind of analysis or modeling is
  performing an exploratory data analysis, or EDA for short.

  \begin{itemize}
  \tightlist
  \item
    Get distributions of the individual variables in your data,
  \item
    Find out any potential relationships exist between variables,
  \item
    Find out any outliers and/or missing values, and
  \item
    (most importantly) helps you to decide how to build your model.
  \end{itemize}
\item
  Here are three common steps in EDA:

  \begin{enumerate}
  \tightlist
  \item
    Examine the raw data values.
  \item
    Compute summary statistics, such as means, medians, and
    interquartile ranges.
  \item
    Create data visualizations.
  \end{enumerate}
\end{itemize}
\end{frame}

\begin{frame}[fragile]{Step 1: Examine the raw data values}
\protect\hypertarget{step-1-examine-the-raw-data-values}{}
\small

\begin{Shaded}
\begin{Highlighting}[]
\NormalTok{evals\_ch5 }\OtherTok{\textless{}{-}}\NormalTok{ evals }\SpecialCharTok{\%\textgreater{}\%}
  \FunctionTok{select}\NormalTok{(ID, score, bty\_avg, age)   }\CommentTok{\# take subset}
\FunctionTok{glimpse}\NormalTok{(evals\_ch5)}
\end{Highlighting}
\end{Shaded}

\begin{verbatim}
Rows: 463
Columns: 4
$ ID      <int> 1, 2, 3, 4, 5, 6, 7, 8, 9, 10, 11, 12, 13, 14, 15, 16, 17, 18,~
$ score   <dbl> 4.7, 4.1, 3.9, 4.8, 4.6, 4.3, 2.8, 4.1, 3.4, 4.5, 3.8, 4.5, 4.~
$ bty_avg <dbl> 5.000, 5.000, 5.000, 5.000, 3.000, 3.000, 3.000, 3.333, 3.333,~
$ age     <int> 36, 36, 36, 36, 59, 59, 59, 51, 51, 40, 40, 40, 40, 40, 40, 40~
\end{verbatim}

\normalsize
\end{frame}

\begin{frame}[fragile]{Step 1: Examine the raw data values}
\protect\hypertarget{step-1-examine-the-raw-data-values-1}{}
An alternative way to look at the raw data values is by choosing a
random sample of the rows.

\small

\begin{Shaded}
\begin{Highlighting}[]
\NormalTok{evals\_ch5 }\SpecialCharTok{\%\textgreater{}\%}
  \FunctionTok{sample\_n}\NormalTok{(}\AttributeTok{size =} \DecValTok{5}\NormalTok{)}
\end{Highlighting}
\end{Shaded}

\begin{verbatim}
# A tibble: 5 x 4
     ID score bty_avg   age
  <int> <dbl>   <dbl> <int>
1   218   4.4    4       42
2   435   3.1    2       62
3    68   4.1    4.83    42
4   227   3.3    8.17    39
5   128   4.3    3       62
\end{verbatim}

\normalsize
\end{frame}

\begin{frame}[fragile]{Step 2: summary statistics}
\protect\hypertarget{step-2-summary-statistics}{}
\small

\begin{Shaded}
\begin{Highlighting}[]
\NormalTok{evals\_ch5 }\SpecialCharTok{\%\textgreater{}\%}
  \FunctionTok{summarize}\NormalTok{(}\AttributeTok{mean\_bty\_avg =} \FunctionTok{mean}\NormalTok{(bty\_avg), }
            \AttributeTok{mean\_score =} \FunctionTok{mean}\NormalTok{(score),}
            \AttributeTok{median\_bty\_avg =} \FunctionTok{median}\NormalTok{(bty\_avg), }
            \AttributeTok{median\_score =} \FunctionTok{median}\NormalTok{(score))}
\end{Highlighting}
\end{Shaded}

\begin{verbatim}
# A tibble: 1 x 4
  mean_bty_avg mean_score median_bty_avg median_score
         <dbl>      <dbl>          <dbl>        <dbl>
1         4.42       4.17           4.33          4.3
\end{verbatim}
\end{frame}

\begin{frame}[fragile]{Step 2: summary statistics}
\protect\hypertarget{step-2-summary-statistics-1}{}
The \texttt{skim()} function from the \texttt{skimr} package, ``skims''
the data, and returns commonly used summary statistics.

\begin{Shaded}
\begin{Highlighting}[]
\FunctionTok{library}\NormalTok{(skimr)}
\NormalTok{evals\_ch5 }\SpecialCharTok{\%\textgreater{}\%} 
  \FunctionTok{select}\NormalTok{(score, bty\_avg) }\SpecialCharTok{\%\textgreater{}\%} 
  \FunctionTok{skim\_without\_charts}\NormalTok{()}
\end{Highlighting}
\end{Shaded}

\begin{longtable}[]{@{}ll@{}}
\caption{Data summary}\tabularnewline
\toprule\noalign{}
\endfirsthead
\endhead
Name & Piped data \\
Number of rows & 463 \\
Number of columns & 2 \\
\_\_\_\_\_\_\_\_\_\_\_\_\_\_\_\_\_\_\_\_\_\_\_ & \\
Column type frequency: & \\
numeric & 2 \\
\_\_\_\_\_\_\_\_\_\_\_\_\_\_\_\_\_\_\_\_\_\_\_\_ & \\
Group variables & None \\
\bottomrule\noalign{}
\end{longtable}

\textbf{Variable type: numeric}

\begin{longtable}[]{@{}
  >{\raggedright\arraybackslash}p{(\columnwidth - 18\tabcolsep) * \real{0.1944}}
  >{\raggedleft\arraybackslash}p{(\columnwidth - 18\tabcolsep) * \real{0.1389}}
  >{\raggedleft\arraybackslash}p{(\columnwidth - 18\tabcolsep) * \real{0.1944}}
  >{\raggedleft\arraybackslash}p{(\columnwidth - 18\tabcolsep) * \real{0.0694}}
  >{\raggedleft\arraybackslash}p{(\columnwidth - 18\tabcolsep) * \real{0.0694}}
  >{\raggedleft\arraybackslash}p{(\columnwidth - 18\tabcolsep) * \real{0.0694}}
  >{\raggedleft\arraybackslash}p{(\columnwidth - 18\tabcolsep) * \real{0.0694}}
  >{\raggedleft\arraybackslash}p{(\columnwidth - 18\tabcolsep) * \real{0.0694}}
  >{\raggedleft\arraybackslash}p{(\columnwidth - 18\tabcolsep) * \real{0.0556}}
  >{\raggedleft\arraybackslash}p{(\columnwidth - 18\tabcolsep) * \real{0.0694}}@{}}
\toprule\noalign{}
\begin{minipage}[b]{\linewidth}\raggedright
skim\_variable
\end{minipage} & \begin{minipage}[b]{\linewidth}\raggedleft
n\_missing
\end{minipage} & \begin{minipage}[b]{\linewidth}\raggedleft
complete\_rate
\end{minipage} & \begin{minipage}[b]{\linewidth}\raggedleft
mean
\end{minipage} & \begin{minipage}[b]{\linewidth}\raggedleft
sd
\end{minipage} & \begin{minipage}[b]{\linewidth}\raggedleft
p0
\end{minipage} & \begin{minipage}[b]{\linewidth}\raggedleft
p25
\end{minipage} & \begin{minipage}[b]{\linewidth}\raggedleft
p50
\end{minipage} & \begin{minipage}[b]{\linewidth}\raggedleft
p75
\end{minipage} & \begin{minipage}[b]{\linewidth}\raggedleft
p100
\end{minipage} \\
\midrule\noalign{}
\endhead
score & 0 & 1 & 4.17 & 0.54 & 2.30 & 3.80 & 4.30 & 4.6 & 5.00 \\
bty\_avg & 0 & 1 & 4.42 & 1.53 & 1.67 & 3.17 & 4.33 & 5.5 & 8.17 \\
\bottomrule\noalign{}
\end{longtable}
\end{frame}

\end{document}
